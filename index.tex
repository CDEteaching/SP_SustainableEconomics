% Options for packages loaded elsewhere
% Options for packages loaded elsewhere
\PassOptionsToPackage{unicode}{hyperref}
\PassOptionsToPackage{hyphens}{url}
\PassOptionsToPackage{dvipsnames,svgnames,x11names}{xcolor}
%
\documentclass[
  a4paper,
  openany]{book}
\usepackage{xcolor}
\usepackage{amsmath,amssymb}
\setcounter{secnumdepth}{3}
\usepackage{iftex}
\ifPDFTeX
  \usepackage[T1]{fontenc}
  \usepackage[utf8]{inputenc}
  \usepackage{textcomp} % provide euro and other symbols
\else % if luatex or xetex
  \usepackage{unicode-math} % this also loads fontspec
  \defaultfontfeatures{Scale=MatchLowercase}
  \defaultfontfeatures[\rmfamily]{Ligatures=TeX,Scale=1}
\fi
\usepackage{lmodern}
\ifPDFTeX\else
  % xetex/luatex font selection
  \setmainfont[]{Times New Roman}
  \setsansfont[]{Arial}
  \setmonofont[]{Courier New}
\fi
% Use upquote if available, for straight quotes in verbatim environments
\IfFileExists{upquote.sty}{\usepackage{upquote}}{}
\IfFileExists{microtype.sty}{% use microtype if available
  \usepackage[]{microtype}
  \UseMicrotypeSet[protrusion]{basicmath} % disable protrusion for tt fonts
}{}
\makeatletter
\@ifundefined{KOMAClassName}{% if non-KOMA class
  \IfFileExists{parskip.sty}{%
    \usepackage{parskip}
  }{% else
    \setlength{\parindent}{0pt}
    \setlength{\parskip}{6pt plus 2pt minus 1pt}}
}{% if KOMA class
  \KOMAoptions{parskip=half}}
\makeatother
% Make \paragraph and \subparagraph free-standing
\makeatletter
\ifx\paragraph\undefined\else
  \let\oldparagraph\paragraph
  \renewcommand{\paragraph}{
    \@ifstar
      \xxxParagraphStar
      \xxxParagraphNoStar
  }
  \newcommand{\xxxParagraphStar}[1]{\oldparagraph*{#1}\mbox{}}
  \newcommand{\xxxParagraphNoStar}[1]{\oldparagraph{#1}\mbox{}}
\fi
\ifx\subparagraph\undefined\else
  \let\oldsubparagraph\subparagraph
  \renewcommand{\subparagraph}{
    \@ifstar
      \xxxSubParagraphStar
      \xxxSubParagraphNoStar
  }
  \newcommand{\xxxSubParagraphStar}[1]{\oldsubparagraph*{#1}\mbox{}}
  \newcommand{\xxxSubParagraphNoStar}[1]{\oldsubparagraph{#1}\mbox{}}
\fi
\makeatother


\usepackage{longtable,booktabs,array}
\usepackage{calc} % for calculating minipage widths
% Correct order of tables after \paragraph or \subparagraph
\usepackage{etoolbox}
\makeatletter
\patchcmd\longtable{\par}{\if@noskipsec\mbox{}\fi\par}{}{}
\makeatother
% Allow footnotes in longtable head/foot
\IfFileExists{footnotehyper.sty}{\usepackage{footnotehyper}}{\usepackage{footnote}}
\makesavenoteenv{longtable}
\usepackage{graphicx}
\makeatletter
\newsavebox\pandoc@box
\newcommand*\pandocbounded[1]{% scales image to fit in text height/width
  \sbox\pandoc@box{#1}%
  \Gscale@div\@tempa{\textheight}{\dimexpr\ht\pandoc@box+\dp\pandoc@box\relax}%
  \Gscale@div\@tempb{\linewidth}{\wd\pandoc@box}%
  \ifdim\@tempb\p@<\@tempa\p@\let\@tempa\@tempb\fi% select the smaller of both
  \ifdim\@tempa\p@<\p@\scalebox{\@tempa}{\usebox\pandoc@box}%
  \else\usebox{\pandoc@box}%
  \fi%
}
% Set default figure placement to htbp
\def\fps@figure{htbp}
\makeatother

\ifLuaTeX
  \usepackage{luacolor}
  \usepackage[soul]{lua-ul}
\else
  \usepackage{soul}
\fi

% definitions for citeproc citations
\NewDocumentCommand\citeproctext{}{}
\NewDocumentCommand\citeproc{mm}{%
  \begingroup\def\citeproctext{#2}\cite{#1}\endgroup}
\makeatletter
 % allow citations to break across lines
 \let\@cite@ofmt\@firstofone
 % avoid brackets around text for \cite:
 \def\@biblabel#1{}
 \def\@cite#1#2{{#1\if@tempswa , #2\fi}}
\makeatother
\newlength{\cslhangindent}
\setlength{\cslhangindent}{1.5em}
\newlength{\csllabelwidth}
\setlength{\csllabelwidth}{3em}
\newenvironment{CSLReferences}[2] % #1 hanging-indent, #2 entry-spacing
 {\begin{list}{}{%
  \setlength{\itemindent}{0pt}
  \setlength{\leftmargin}{0pt}
  \setlength{\parsep}{0pt}
  % turn on hanging indent if param 1 is 1
  \ifodd #1
   \setlength{\leftmargin}{\cslhangindent}
   \setlength{\itemindent}{-1\cslhangindent}
  \fi
  % set entry spacing
  \setlength{\itemsep}{#2\baselineskip}}}
 {\end{list}}
\usepackage{calc}
\newcommand{\CSLBlock}[1]{\hfill\break\parbox[t]{\linewidth}{\strut\ignorespaces#1\strut}}
\newcommand{\CSLLeftMargin}[1]{\parbox[t]{\csllabelwidth}{\strut#1\strut}}
\newcommand{\CSLRightInline}[1]{\parbox[t]{\linewidth - \csllabelwidth}{\strut#1\strut}}
\newcommand{\CSLIndent}[1]{\hspace{\cslhangindent}#1}



\setlength{\emergencystretch}{3em} % prevent overfull lines

\providecommand{\tightlist}{%
  \setlength{\itemsep}{0pt}\setlength{\parskip}{0pt}}



 


\makeatletter
\@ifpackageloaded{tcolorbox}{}{\usepackage[skins,breakable]{tcolorbox}}
\@ifpackageloaded{fontawesome5}{}{\usepackage{fontawesome5}}
\definecolor{quarto-callout-color}{HTML}{909090}
\definecolor{quarto-callout-note-color}{HTML}{0758E5}
\definecolor{quarto-callout-important-color}{HTML}{CC1914}
\definecolor{quarto-callout-warning-color}{HTML}{EB9113}
\definecolor{quarto-callout-tip-color}{HTML}{00A047}
\definecolor{quarto-callout-caution-color}{HTML}{FC5300}
\definecolor{quarto-callout-color-frame}{HTML}{acacac}
\definecolor{quarto-callout-note-color-frame}{HTML}{4582ec}
\definecolor{quarto-callout-important-color-frame}{HTML}{d9534f}
\definecolor{quarto-callout-warning-color-frame}{HTML}{f0ad4e}
\definecolor{quarto-callout-tip-color-frame}{HTML}{02b875}
\definecolor{quarto-callout-caution-color-frame}{HTML}{fd7e14}
\makeatother
\makeatletter
\@ifpackageloaded{bookmark}{}{\usepackage{bookmark}}
\makeatother
\makeatletter
\@ifpackageloaded{caption}{}{\usepackage{caption}}
\AtBeginDocument{%
\ifdefined\contentsname
  \renewcommand*\contentsname{Table of contents}
\else
  \newcommand\contentsname{Table of contents}
\fi
\ifdefined\listfigurename
  \renewcommand*\listfigurename{List of Figures}
\else
  \newcommand\listfigurename{List of Figures}
\fi
\ifdefined\listtablename
  \renewcommand*\listtablename{List of Tables}
\else
  \newcommand\listtablename{List of Tables}
\fi
\ifdefined\figurename
  \renewcommand*\figurename{Figure}
\else
  \newcommand\figurename{Figure}
\fi
\ifdefined\tablename
  \renewcommand*\tablename{Table}
\else
  \newcommand\tablename{Table}
\fi
}
\@ifpackageloaded{float}{}{\usepackage{float}}
\floatstyle{ruled}
\@ifundefined{c@chapter}{\newfloat{codelisting}{h}{lop}}{\newfloat{codelisting}{h}{lop}[chapter]}
\floatname{codelisting}{Listing}
\newcommand*\listoflistings{\listof{codelisting}{List of Listings}}
\makeatother
\makeatletter
\makeatother
\makeatletter
\@ifpackageloaded{caption}{}{\usepackage{caption}}
\@ifpackageloaded{subcaption}{}{\usepackage{subcaption}}
\makeatother
\usepackage{bookmark}
\IfFileExists{xurl.sty}{\usepackage{xurl}}{} % add URL line breaks if available
\urlstyle{same}
\hypersetup{
  pdftitle={Introduction to Sustainable Economics},
  colorlinks=true,
  linkcolor={Maroon},
  filecolor={Maroon},
  citecolor={Blue},
  urlcolor={blue},
  pdfcreator={LaTeX via pandoc}}


\title{Introduction to Sustainable Economics}
\usepackage{etoolbox}
\makeatletter
\providecommand{\subtitle}[1]{% add subtitle to \maketitle
  \apptocmd{\@title}{\par {\large #1 \par}}{}{}
}
\makeatother
\subtitle{Additional attainments for the Master's in Sustainability
Transformations}
\author{}
\date{}
\begin{document}
\frontmatter
\maketitle

\renewcommand*\contentsname{Table of contents}
{
\hypersetup{linkcolor=}
\setcounter{tocdepth}{2}
\tableofcontents
}

\mainmatter
\bookmarksetup{startatroot}

\chapter*{Preface}\label{preface}
\addcontentsline{toc}{chapter}{Preface}

\markboth{Preface}{Preface}

\emph{Version 1.0, October 2025}

This e-learning course is intended for students to work through it at
their own pace. There are no sessions held in presence. However,
students can reach out to the lecturer for questions and support at any
time via email. The course is structured along six sections that build
on each other. Many sections are interactive and contain links and
videos to consult. For more information on the course please see the
information document of the course.

The course is divided into three blocks. In the introductory block,
students are introduced to basic concepts and the theoretical
assumptions of economics (Section 1-4). This is followed by a block on
the problem analysis addressing the ecological, social, and economic
challenges faced by the current economic and societal system (Section
5). In the final block, participants explore various building blocks
that provide starting points for shaping a sustainable economy (Section
6-7).

Below you will find the contents of the course. Following the script
will guide you through all the sections and learning material in a
sensible order. Since this course requires a high degree of autonomy in
acquiring the material, we also encourage you to contact us via email if
you want to discuss questions (see information document).

\textbf{Learning Outcomes}

Students can:

\begin{itemize}
\tightlist
\item
  understand that there are different theories in economics and are able
  to name them.
\item
  can identify the foundations of the emerging field of sustainable
  economics.
\item
  can distinguish between the problem analyses and solution approaches
  of sustainable economics and those of today's dominant school of
  thought (neoclassical and environmental economics).
\item
  are able to critically question both traditional and new economic
  approaches (concepts) and assess the conflicts inherent in them.
\item
  can reflect on the strengths and weaknesses of different theoretical
  approaches, both in general and in relation to concrete problems.
\item
  can analyse how alternative economic models (e.g., circular economy,
  post-growth economics) function and evaluate their potential and
  limitations for economic policy.
\item
  acquire basic knowledge about measures and transformation scenarios
  for implementing alternative economic models beyond the focus on
  maximizing profits and goods production.
\end{itemize}

\textbf{Assessment}

The Assessment is based on two parts.

The first part is a quiz. The quiz covers the basic concepts that are
mainly discussed in section 3.
\href{https://ilias.unibe.ch/go/tst/3512947}{The quiz is done on ilias}.
You will have two attempts to complete the quiz. The better attempt will
count. You need 50\% of the points to pass the quiz. You have 15 minutes
to complete the quiz.

The second part of this assessment requires you to write a reflective
essay on the material you have studied and your understanding of
economics and the economy. We invite you to consider how the course
material has (or hasn't) changed your understanding of economics and the
economy in the context of sustainable transformation. Use the questions
below to guide your reflection paper. However, you can also refer to
other sources if you find them relevant for your reflection. You may use
any resources. Students must sign a declaration of authorship. The
written paper should comprise 6,000--7,500 characters (excluding spaces,
the reference list, the title page, footnotes, headers and footers,
lists, tables, figures and appendices). This corresponds to
approximately three pages.

Guiding questions

\begin{itemize}
\item
  How has your understanding of economics changed as a result of taking
  this course? What implications does this have for your understanding
  of sustainable transformation?
\item
  How has your understanding of the economy changed as a result of this
  course? What are the implications for your understanding of
  sustainable transformation?
\item
  What role does an understanding of economics play in the context of a
  sustainable transformation?
\end{itemize}

\href{https://ilias.unibe.ch/go/exc/3517790}{Please submit your
reflection paper via ilias.}

Both parts of the assessment can be done and submitted at any time. The
course counts as completed as soon as students have worked through all
sections and passed both parts of the assessment. The course is graded
with pass or fail.

Please note that the online exam and the reflection paper must be
completed \textbf{by 30 May 2026 at the latest}.

\textbf{Reading guide}

\begin{tcolorbox}[enhanced jigsaw, breakable, colbacktitle=quarto-callout-note-color!10!white, left=2mm, bottomtitle=1mm, colback=white, rightrule=.15mm, colframe=quarto-callout-note-color-frame, bottomrule=.15mm, opacityback=0, titlerule=0mm, opacitybacktitle=0.6, toptitle=1mm, toprule=.15mm, title={Definitions and further readings}, arc=.35mm, leftrule=.75mm, coltitle=black]

\end{tcolorbox}

\begin{tcolorbox}[enhanced jigsaw, breakable, colbacktitle=quarto-callout-tip-color!10!white, left=2mm, bottomtitle=1mm, colback=white, rightrule=.15mm, colframe=quarto-callout-tip-color-frame, bottomrule=.15mm, opacityback=0, titlerule=0mm, opacitybacktitle=0.6, toptitle=1mm, toprule=.15mm, title={Examples and reflections}, arc=.35mm, leftrule=.75mm, coltitle=black]

\end{tcolorbox}

\begin{tcolorbox}[enhanced jigsaw, breakable, colbacktitle=quarto-callout-caution-color!10!white, left=2mm, bottomtitle=1mm, colback=white, rightrule=.15mm, colframe=quarto-callout-caution-color-frame, bottomrule=.15mm, opacityback=0, titlerule=0mm, opacitybacktitle=0.6, toptitle=1mm, toprule=.15mm, title={Exercises and links to ILIAS}, arc=.35mm, leftrule=.75mm, coltitle=black]

\end{tcolorbox}

\bookmarksetup{startatroot}

\chapter{Introduction}\label{introduction}

\textbf{Learning Outcomes}

\begin{itemize}
\tightlist
\item
  can explain what economics is concerned with.
\end{itemize}

\subsection{The discipline of
economics}\label{the-discipline-of-economics}

The term `economics' helps to situate the discipline within the broader
landscape of academic fields (see Fig. 1). As part of the humanities and
cultural sciences, economic science deals explicitly with the economy as
a man-made object of study (Engelkamp and Sell (2013), p.~3). Economic
science can be divided into two broad areas: \textbf{economics} and
\textbf{business administration}.

\begin{figure}[H]

\centering{

\pandocbounded{\includegraphics[keepaspectratio]{images/Fig1_1_EconomicScience.png}}

}

\caption{\label{fig-EconScience}Disciplinary positioning of economics.
Source: Own illustration based on Engelkamp \& Sell (2013), p.3.}

\end{figure}%

\textbf{Business administration:} ``Business administration is the study
of the economic, organizational, technical, and financial processes in
companies and various economic institutions'' (Friedli, Müller Vasquez
Callo, and Balmer-Zahnd (2019), p.~16, own translation). In this field,
researchers develop a microeconomic perspective on the respective
subject of investigation.

\textbf{Economics:} Unlike business administration, economics does not
examine what happens within economic actors but focuses on the
interaction between the various economic actors and also takes a
macroeconomic perspective. In this course, we will deal with this part
of economic science.

With the development and differentiation of economic theories, various
definitions of economics have been established and advocated. The most
widely accepted definition within economics today was originally
proposed by Lionel Robbins and emphasizes the relationship between the
\textbf{scarcity of resources} and the satisfaction of \textbf{needs} by
\textbf{economic agents}. Robbins defines economics as follows:
``\emph{Economics is the science which studies human behaviour as a
relationship between ends and scarce means which have alternative
uses}'' (Robbins (1932), p.~15). Robbins' definition is influenced by
his theoretical orientation along the \textbf{neoclassical school of
thought} (mainstream/orthodox economics). Building on this, most
mainstream textbooks define economics as a science of (rational)
decision-making that deals with how people make decision in the light of
scarce resources to achieve their goals.

The heterodox economist Ha-Joon Chang
\href{http://www.huffpost.com/entry/ha-joon-chang-economics_n_5120030}{criticizes}
this definition as too specific and argues that it stems from a
theoretical approach and thus already prescribes a certain approach
(Chang (2014)). Chang defines economics not by its theoretical approach,
but by its \textbf{object of study:} the economy. Economics therefore
deals with everything related to the (re)production, exchange, and
distribution of goods and services to satisfy human needs. In addition
to and between these definitions, there are many other definitions of
economics. Based on different understandings of economics,
\textbf{different schools of thought} consequently focus on different
subject areas, such as power structures, institutions, or macroeconomic
and social relationships and structures. In the section on pluralist
economics (section 4), we will take a closer look at what constitutes
neoclassical economics (mainstream) and learn more about some of the
other schools of thought. In the following, we will generally follow
Chang's understanding, which takes the subject of investigation and not
a theoretical approach as the basis for our understanding of economics.
Accordingly, this understanding also allows for many different
approaches and schools of thought to be used to understand economic
relationships.

\section{Subfield of economics}\label{subfield-of-economics}

Economics can be divided into \textbf{three major subfields}: economic
theory, economic policy, and public finance. There are different ways to
divide the field of economics, but using these three categories is a
helpful way to get an overview. For our purposes, the first two ---
economic theory and economic policy --- are especially important.

\hfill\break
\textbf{Economic theory} describes theories from the past to the present
and highlights their implications for how the economy is described and
evaluated. There are many different schools of thought within economic
theory (these are discussed in more detail in the learning sequence
\href{https://ilias.unibe.ch/goto.php?target=lso_2808206}{on pluralist
economics}). In principle, all schools of thought can be divided into
two sub-areas of economic theory: microeconomics and macroeconomics.

\begin{itemize}
\item
  \textbf{Microeconomics} deals with individual economic entities, such
  as households and companies, and their mutual relationships. It does
  not make any statements about aggregate economic relationships but
  rather examines the behavior of individual companies and households
  and their coordination process via the market with the help of price
  adjustments (Bontrup and Marquardt (2021), p.~1).
\item
  In contrast, \textbf{Macroeconomics} deals with the aggregate economic
  relationships between economic sectors and different markets and
  examines macroeconomic phenomena. For example, it examines
  macroeconomic indicators such as the unemployment rate or inflation.
\end{itemize}

\textbf{Public finance} examines the various components of the national
budget, such as taxes and subsidies, and their interrelationships. This
implicitly raises questions about the role of the state in resource
allocation and the distribution of income and wealth.

\textbf{Economic policy} deals with state intervention in the economy.
It is concerned with shaping social life in order to increase social
welfare (Bontrup and Marquardt (2021), p.~2). However, views on what
constitutes social welfare can vary greatly. The state has a significant
influence on how economic actors interact, for example, but not
exclusively through targeted interventions. The targets of economic
policy can include employment, price stability, foreign trade, economic
growth, the distribution of income or wealth, and the protection of the
natural environment. Examples of economic policy interventions include
the prohibition of cartels or the introduction of a minimum wage.

\textbf{Economic history} is another important subfield, focusing on
economic policy and dynamics over extended timeframes or through a
historical lens. One possible subject of investigation could be the
effectiveness of economic policy during the Great Depression. The
historical development of economic theories is examined in the context
of the history of economic thought.

In addition, there are various specialized fields within economics, such
as behavioral economics, education economics, and health economics, each
focusing on a specific area of inquiry.

\begin{tcolorbox}[enhanced jigsaw, breakable, colbacktitle=quarto-callout-note-color!10!white, left=2mm, bottomtitle=1mm, colback=white, rightrule=.15mm, colframe=quarto-callout-note-color-frame, bottomrule=.15mm, opacityback=0, titlerule=0mm, opacitybacktitle=0.6, toptitle=1mm, toprule=.15mm, title={Further readings}, arc=.35mm, leftrule=.75mm, coltitle=black]

Ha-Joon Chang's book ``Economics: The User's Guide'' offers an easy
introduction to topics in economics, provides a historical perspective
and a good overview of past and current debates. Depending on your
interests, individual chapters can also be consulted.

Chang, Ha-Joon. 2014. Economics: The User's Guide. First U.S. edition.
New York: Bloomsbury Press.

\end{tcolorbox}

\section{Economics and value
judgments}\label{economics-and-value-judgments}

Many economists --- perhaps even the majority --- hold that economic
theory should be free from assumptions about human behavior, societal
structures, and normative values. Economic theory should therefore be
neutral and free of values and capable of formulating universally valid
laws. According to this view, economics is regarded as a positive
science. A positive science describes the world as it is, without making
value judgments or taking a particular perspective. Typically, natural
sciences are understood as positive sciences. Accordingly, mainstream
economics distinguishes between positive statements (how the world is)
and normative statements (how the world should be) and argues that these
spheres are clearly separable (the fundamental philosophical position
lies in logical positivism). The focus is accordingly on positive
statements.

Critics argue that such a clear distinction between positive and
normative statements, between facts and values, is not possible for
social sciences such as economics and therefore makes objective and
value-free analysis impossible (see, for example, J. B. Davis (2016)).
Hilary Putnam argues that facts and values are often not clearly
separable but intertwined (Putnam (2002)). Thus, economic concepts and
theories are always shaped by values and worldviews in their
construction.

Following the critics we hold it to be difficult --- if not impossible
--- to draw a clear line between positive and normative perspectives.
The different ways in which economics can be defined show that value
judgments are always implicit. One perspective may focus on the
efficient allocation of scarce re-sources in a society, while another
may focus on overcoming poverty and meeting the basic needs of all
people in an economy. As soon as I favor one perspective over another, I
have made an initial value judgment. In section 4, we take a closer look
at the implicit value judgments of neoclassical economics, which sees
itself as positive economics.

Acknowledging that economics is not entirely value-free or objective
does not mean rejecting the existence of facts or treating all
viewpoints as equally valid. Nor does it deny the existence of an
objective reality --- it simply recognizes that we can only approximate
it to varying degrees.

Like any other science, economics must be guided by scientific
principles and methods. In addition, assumptions and value judgments
should be made explicit and transparent. In addition to criticism of the
implicit value judgments behind neoclassical economics, the explicit
framework of assumptions also leads to controversial discussions,
particularly regarding the extent to which the assumptions must
realistically reflect reality. Must the initial assumptions, such as the
existence of perfect competition, be realistic in order to gain valuable
insights? At first glance, the answer is clear: unrealistic or
incomplete assumptions must also lead to false conclusions. According to
the Paul Samuelson (1915-2009), one of the most famous economists of the
20th century, it is impossible to draw true conclusions from
demonstrably false assumptions. However, proponents of so-called
instrumentalism argue differently. They say that, given the complexity
of human behavior, it is necessary to abstract and leave out the
insignificant in analysis. In his search for an efficient method to
discover far-reaching insights and recommendations for action, US
economist Milton Friedman, also one of the most famous economists of the
20th century, even takes an extreme anti-realist position. Working
within the framework of a superficial world is declared an unproblematic
necessity:

\begin{quote}
``Truly important and significant hypotheses will be found to have
assumptions that are widely inaccurate descriptive representations of
reality, and in general the more significant the theory, the more
unrealistic the assumption. {[}\ldots{]} To be important, therefore, a
hypothesis must be descriptively false in its assumptions.'' - Milton
Friedman, 1953, p.~14.
\end{quote}

The debate as to whether economics is free of value judgements or not
has been ongoing since the development of mainstream economics,
neoclassical economics, towards the end of the 19th century and
continues to this day, with no conclusive resolution. This is
exemplified by two quotes from two well-known economists of the first
half of the 20th century:

\begin{quote}
``Economics deals with ascertainable facts; ethics with valuations and
obligations. The two fields of inquiry are not on the same plane of
discourse.'' - Lionel Robbins in An Essay on the Nature and Significance
of Economic Science (1932), p.~132.
\end{quote}

\begin{quote}
``As against Robbins, economics is essentially a moral science. That is
to say, it employs introspection and judgment of value.'' --- John M.
Keynes in a letter to Sir Roy Harrod, July 4, 1938, in Atkinson (2009),
p.~791.
\end{quote}

\subsection{What does this mean for economic
policy?}\label{what-does-this-mean-for-economic-policy}

Economic conclusions --- like economic analyses --- can never be made
entirely free of value judgments. For example, when certain processes
produce gains for some economic actors and losses for others, it is
impossible to assess the overall net effect without introducing
normative assumptions. Even appealing to pure economic efficiency
already reflects a value judgment: it prioritizes efficiency over other
possible criteria for evaluating outcomes. At first glance, it might
seem reasonable to treat gains and losses equally --- adding them up on
a one-to-one basis without weighting. However, this approach assumes
that all outcomes are of equal significance to those affected. In
practice, this is rarely the case. Some individuals may experience
losses that push them into an existential crisis---for instance, if
their income drops below the threshold of physical or cultural
subsistence. Depending on one's ethical perspective, there may be strong
reasons to argue that such losses should be weighted more heavily than
the corresponding gains of others. Not applying any weighting at all is
itself a normative choice. If we allow weights to vary continuously,
then in principle, \textbf{there are infinitely many possible normative
positions} one can take when evaluating such outcomes.

When assessing such market processes, \textbf{normative questions} arise
that economics, like any other social science with its empirical
methods, is unable to answer objectively and value-free. It must
therefore present these questions in such a way that they can be
discussed by society and, if necessary, decided by parliaments through
legislation.

\section{Thought styles and thought
collectives}\label{thought-styles-and-thought-collectives}

As discussed in the previous chapter, economic theories are always
shaped by value judgments and underlying worldviews. Therefore, there is
no clearly objective economic theory, but rather a multitude of
perspectives that make it possible to analyze and describe economic
phenomena. There is no neutral ``view from nowhere'' or purely objective
perspective, as our perception is shaped by numerous influences, such as
our education, social and cultural environment, language, material
living conditions, and personal experiences. These influences shape our
perspective of the world. A helpful metaphor is that of glasses: some
lenses allow us to see broad patterns, while others reveal fine details
and nuances. These different ``glasses'' influence how we see the world
--- not arbitrarily, but within identifiable limits. As discussed
earlier, multiperspectivity has boundaries; not every interpretation is
equally valid or supported by evidence.

In the philosophy of science, such perspectives are often referred to as
paradigms or thought styles. The concept of ``thought style'' (Denkstil)
was developed by Ludwik Fleck, a Polish physician and philosopher of
science, decades before Thomas Kuhn introduced the more widely known
idea of paradigms. Fleck, born in 1896 in Poland and died in 1961 in
Israel. In 1944 he was deported to the concentration camp Buchenwald. He
was tasked with developing a typhus vaccine for the SS. However, he
reportedly administered placebos to the guards and gave the actual
medicine to fellow prisoners.

In his scientific work, Fleck emphasized the dependence of thinking on
values and contexts. He rejected the idea that knowledge is objective,
neutral, and universal. He argued that the production and use of
knowledge take place in specific environments, which influences its
meaning and effect. Knowledge is used differently in concentration camps
than in NGOs or corporate research departments. Knowledge is always
embedded in institutions and power structures, and scientists are
influenced by their environment and previous experiences.

Thought collectives are groups of people who share a common way of
thinking and use certain concepts and methods. Thought collectives
therefore use a specific lens through which to view the world. These
thought collectives are often conservative and resistant to change,
i.e., they resist changes and further developments in their way of
thinking. Since, as Fleck emphasized, people can think, argue, and
understand in fundamentally different ways, people who belong to
different thought collectives often have difficulty understanding the
thought processes of other thought styles.

Fleck's findings show that science and reasoning are by no means
independent of social, cultural, and historical contexts. They
illustrate the diversity of perspectives from which scientific problems
can be viewed and emphasize that no single theory or way of thinking can
fully capture the whole of reality. Instead, different thought styles
and thought collectives are capable of providing diverse insights and
enriching our understanding of complex phenomena. This leads to
different economic policy recommendations and offers a wide range of
approaches to solving complex challenges.

\subsection{Thought collective in
economics}\label{thought-collective-in-economics}

As already indicated several times, economics also consists of a
multitude of different thought collectives that adopt certain
perspectives or ways of thinking based on value judgments and
worldviews. This diversity of thought collectives (often referred to as
schools of thought) is described by the term ``pluralist economics.'' At
present, economics is strongly dominated by one thought style known as
neoclassical economics, which is often described as mainstream
economics. However, we believe that students should not only learn about
the thought style of the prevailing thought collective but also gain an
insight into how thought styles in economics have developed and what
other thought styles exist in economics. The next chapter provides an
introduction to the history and development of thought styles in
economics. We will then deal with current schools of thought in the
following sections.

\begin{tcolorbox}[enhanced jigsaw, breakable, colbacktitle=quarto-callout-note-color!10!white, left=2mm, bottomtitle=1mm, colback=white, rightrule=.15mm, colframe=quarto-callout-note-color-frame, bottomrule=.15mm, opacityback=0, titlerule=0mm, opacitybacktitle=0.6, toptitle=1mm, toprule=.15mm, title={Further readings}, arc=.35mm, leftrule=.75mm, coltitle=black]

\begin{itemize}
\item
  \href{https://ilias.unibe.ch/go/file/3512752}{Yuval Harari and the
  legend of Peugeot} in his book ``A brief history of humankind'' His
  thesis: The truly sufficient difference between us and other animals
  is our ability to create cooperative net-works in which millions of
  complete strangers work together toward common goals. We can
  co-operate on a large scale because we follow shared intersubjective
  fictions. We work successfully with strangers because, like them, we
  believe in things such as gods, nations, money, or human rights. And
  yet none of these things exist outside the stories that humans invent
  and tell each other. There are no gods in the universe, no nations, no
  money, and no human rights---except in the collective imagination of
  humans.
\item
  \href{https://pitchforkeconomics.com/episode/what-is-the-trick-in-trickle-down/}{Podcast
  with Yuval Harari on the topic of ``trickle-down economics'': What is
  the trick in trickle down?\\
  }It's how wealthy elites and their neoliberal lackeys convince you
  that what's good for them (tax cuts, deregulation, etc.) is good for
  you and that policies like the minimum wage, overtime, and paid sick
  leave will ruin the economy. Economics is a story we tell ourselves to
  help explain who gets what, and why. In this episode, we explore how
  to tell a better story.
\end{itemize}

\end{tcolorbox}

\bookmarksetup{startatroot}

\chapter{History of economic thought}\label{history-of-economic-thought}

\textbf{Learning Outcome}

Students\ldots{}

\begin{itemize}
\tightlist
\item
  can explain what the term ``social physics'' means. - recognize the
  implications of applying social-physical principles in economics.
\end{itemize}

Walter Ötsch is a professor at the Koblenz University of Applied
Sciences. As an economist and cultural historian, he researches and
publishes on socio-political and economic topics such as populism and
the social role of markets. Below you will find a section (chapter 4-9)
of a working paper written by him.

\begin{tcolorbox}[enhanced jigsaw, breakable, colbacktitle=quarto-callout-caution-color!10!white, left=2mm, bottomtitle=1mm, colback=white, rightrule=.15mm, colframe=quarto-callout-caution-color-frame, bottomrule=.15mm, opacityback=0, titlerule=0mm, opacitybacktitle=0.6, toptitle=1mm, toprule=.15mm, title={Ötsch (2018) - Download and read at least chapters 4-9:}, arc=.35mm, leftrule=.75mm, coltitle=black]

In this working paper Walter Ötsch describes how economics developed
from the moral science of Adam Smith to a science with a biologically
determined view of human nature under Malthus and Ricardo. In this
process, scientific metaphors (clock system, balance scale, computer
information) became increasingly important and economics is increasingly
seen as ``social physics''. This process is very illustrative for the
development of modern neoclassical economics.

\href{https://ilias.unibe.ch/go/file/3524254}{Download and read Ötsch
(2018) - at least chapters 4-9}

\end{tcolorbox}

\begin{tcolorbox}[enhanced jigsaw, breakable, colbacktitle=quarto-callout-note-color!10!white, left=2mm, bottomtitle=1mm, colback=white, rightrule=.15mm, colframe=quarto-callout-note-color-frame, bottomrule=.15mm, opacityback=0, titlerule=0mm, opacitybacktitle=0.6, toptitle=1mm, toprule=.15mm, title={Brief digression: History of Homo Economicus (optional for interested
readers)}, arc=.35mm, leftrule=.75mm, coltitle=black]

The history of homo economicus dates back at least to the mention of
``economicus'' in Xenophon's work of the same name in the fourth century
BC (Wilson and Dixon (2014), p.~11). In the long period leading up to
the establishment of modern economics by Adam Smith in the 18th century,
various conceptions of economic man were used\hyperref[_ftn1]{{[}1{]}} ,
often strongly influenced by the corresponding ``Zeitgeist''. The
concept of homo economicus as a rational utility maximizer is often
attributed to Smith, as he recognized self-love as a central
characteristic of human behavior. However, as is often the case with
Smith's concepts, this common interpretation is oversimplified. In
addition to self-love, Adam Smith identifies other human characteristics
that are central to human behavior. Smith paints a much more complex
picture of human behavior than the reduced version of modern homo
economicus (Hill (2012)). Smith's dense description of economic man
makes it impossible to model him or work with him mathematically (Morgan
(2006); Morgan (2012)). This was changed over time through a conscious
reduction of the description of the economic man.

\hyperref[_ftnref1]{{[}1{]}} In this text, the term \emph{``economic
man}'' is used when referring to the idea of an economic agent. This is
because the term ``homo economicus'' represents the currently dominant
conception and ``economic man'' is used in this sense in (predominantly
English) literature. The term is problematic in that it implies the
exclusion of an ``economic woman''. However, if we consider the
orientation of economics today and also throughout history, such a
representation of economics is unfortunately not too far-fetched as
feminist economists have argued many times.

At the beginning of the 19th century, Thomas Robert Malthus and David
Ricardo were already using somewhat reduced versions of economic man,
which already had certain model characteristics (Morgan (1996),Morgan
(2006), Morgan (2012)). John Stuart Mill took this further and reduced
economic man to the characteristics that were central to the economic
sphere: the desire to accumulate wealth, the aversion to work, and the
desire for luxury goods (Mill (1996)). Although the model of economic
man underwent a reduction in Malthus, Ricardo, and Mill, the model of
economic man cannot yet be used in the same way as it is today.

William Stanley Jevons and Francis Edgeworth laid the foundations for
this development at the end of the 19th century. Alongside Alfred
Marshall, Léon Walras, and Vilfredo Pareto, they are considered
important founders of neoclassical economics and were part of the
\textbf{marginalist revolution} (this will be discussed in more detail
in section 3 on basic concepts of economics; see the marginal
principle). Like some other economists with a background in mathematics
or physics, the two economists were determined to turn economics into a
natural science. Neoclassical economists were convinced that the
analytical and deductive method should also be the ideal approach for
economics. Jevons justified the necessity of mathematics for economics
simply by pointing out that it deals with quantities (Jevons (1879),
p.~4). This group of economists was guided by physics and physical
concepts. Irving Fisher, another of the first neoclassical economists,
explicitly used various mechanical or hydraulic concepts for his
theories and applied them to economics.

\begin{figure}[H]

\centering{

\pandocbounded{\includegraphics[keepaspectratio]{images/Fig1_2_fisherA.png}}

}

\caption{\label{fig-fisherA}Fisher's mechanical balance of exchange to
illustrate the quantity theory of money (Fischer 1922,p.21)}

\end{figure}%

\begin{figure}[H]

\centering{

\pandocbounded{\includegraphics[keepaspectratio]{images/Fig1_3_fisherB.png}}

}

\caption{\label{fig-fisherB}Fisher uses a hydraulic model to illustrate
purchasing power in a bimetallism system (Fisher 1922, p.~119)}

\end{figure}%

The idea of \textbf{equilibrium} itself is a concept borrowed from
physics. This fixation on physics is also described as ``physics envy'',
and Mirowski shows that this orientation toward physics was a
fundamental driver in the development of the history of economic thought
(Mirowski (1989), p.~396). Even well into the 20th century, the explicit
analogy to physics remained in economics. A good example of this, and
also of the basic toolbox of economists, is the
\href{https://www.youtube.com/watch?v=rAZavOcEnLg}{machine} built by
William Phillips in the 1940s to represent the British economy. The
development of homo economicus is also exemplary of the mechanistic
worldview that emerged in neoclassical economics, as described by Walter
Ötsch in his text.

Among other things, the complexity of economic man stood in the way of
the applicability of mathematics to economics. Jevons therefore
attempted to reduce this complexity. Influenced by Bentham's
utilitarianism, Jevons reduced the seven criteria formulated by Bentham,
which are central to the calculation of utility and thus influence human
behavior, to just two, arguing that only these were economically
relevant (Jevons (1879), p.~17). This enabled Jevons to work
mathematically with economic man (Reiss (2000)). This abstraction and
objectification of economics and economic man is most clearly
illustrated by the concepts developed by Francis Endgeworth and later
Frank Knight. Endgeworth refers to economic man as a ``pleasure
machine'' and Knight as a ``slot machine'' (Morgan (2006)). Pareto's use
of his ``homo economicus'' also establishes an \textbf{analysis of
humans from a mechanical perspective.} Pareto writes: ``Just as
analytical mechanics deals with material points and rigid bodies,
mathematical economics considers an abstract human being, a homo
economicus'' (Pareto (1902), p.~1100).

With the marginalist revolution, homo economicus became the focus of
economics, in stark contrast to classical economists (Smith, Malthus,
Ricardo, Marx), who focused primarily on class analysis. In addition,
abstraction now enabled modeling and mathematical applications, which
was intended to bring economics closer to the natural sciences. The
rational and utility-maximizing agent became the basis of economics. The
underlying \textbf{rational choice theory} was subsequently further
developed and adapted (e.g., through the concept of bounded rationality)
but remains fundamentally the same to this day.

The history behind this development shows that it was not a natural one,
but rather a conscious effort to push \textbf{a certain direction in
economics}. For example, the neoclassical approach makes it impossible
to identify and analyze structural problems, as the focus is on marginal
changes in certain indicators (e.g., inflation or interest rates). A
good example of this is the work of the winners of the 2019 Alfred Nobel
Memorial Prize in Economics, Abhijit Banerjee, Esther Duflo, and Michale
Kremer. In their research on poverty reduction, they use experiments
(randomized control trials) to investigate the impact of specific
individual measures to combat poverty. The analysis (and thus also a
change) of the structural causes of these people's poverty is lost sight
of here.

The descriped development in economics was heavily criticized when it
first emerged. Criticism of its focus, its fixation on analytical models
and its strong abstraction from the real world continues to this day. In
view of the weaknesses that have been increasingly exposed in
neoclassical economics in the years since the 2008 financial crisis, the
concerns of Alfred Marshall -- himself a famous neoclassical economist
-- about this development are once again becoming relevant. In 1881, in
a commentary on Edgeworth's ``Mathematical Psychics,'' Marshall wondered
whether Edgeworth would succeed in preventing mathematics from running
away with him and driving him out of sight of the actual facts of
economics (Vazquez (1995), p.~251).

\end{tcolorbox}

\begin{tcolorbox}[enhanced jigsaw, breakable, colbacktitle=quarto-callout-note-color!10!white, left=2mm, bottomtitle=1mm, colback=white, rightrule=.15mm, colframe=quarto-callout-note-color-frame, bottomrule=.15mm, opacityback=0, titlerule=0mm, opacitybacktitle=0.6, toptitle=1mm, toprule=.15mm, title={Further reading}, arc=.35mm, leftrule=.75mm, coltitle=black]

The following further reading is recommended for students who would like
to delve deeper and engage more intensively with the history of
theory:\\

\begin{itemize}
\item
  Ambler, Lucy, Joe Earle, und Nicola Scott. 2022. «Whitewashes
  History». In Reclaiming Economics for Future Generations, Manchester
  Capitalism Ser, Manchester: Manchester University Press, 118--60.
\item
  McCloskey, Deirdre N. 1998. The Rhetoric of Economics. 2nd
  ed.~Madison, Wis: University of Wisconsin Press.
\item
  Mirowski, Philip. 1989. More Heat than Light: Economics as Social
  Physics, Physics as Nature's Economics. Cam-bridge: Cambridge
  University Press. doi:10.1017/CBO9780511559990
\end{itemize}

\end{tcolorbox}

\bookmarksetup{startatroot}

\chapter{Basic concepts of economics}\label{basic-concepts-of-economics}

This module deals with the \textbf{basic concepts of economics}. It is
intended as an introduction to the central concepts and topics of
economics. We are aware that many of these terms are new to students
without a background in economics. Thus, the aim of this module is not
to provide you with detailed knowledge of all topics, but rather
\textbf{to give} you an \textbf{initial overview of economics.} For
students of economics, this chapter serves as a review and
contextualization of existing knowledge.

Each basic concept is accompanied by a number \textbf{of in-depth
questions}. These serve to allow you to test your knowledge of the
respective area independently but are also intended to stimulate thought
and reflection. If you have any questions about the content or would
like to explore a topic in greater depth, you can contact us (see
information document).

By the end of this volume you will find a link to the quiz on the basic
concepts of economics. This quiz is part of the assessment and will
account for 40\% of your final grade. Consult the information document
for more information.

\textbf{Learning Outcome}

Students can\ldots{}

\begin{itemize}
\item
  explain the basic concepts of economics.
\item
  explain the fundamentals of pluralist economics and identify different
  schools of thought.
\end{itemize}

\section{Forms of provisioning in the
economy}\label{forms-of-provisioning-in-the-economy}

At the latest with the division of labor, questions arise for an
economic system about the organization of (re)production, consumption,
and distribution. This raises the following fundamental questions:

\begin{itemize}
\item
  How is social surplus distributed?
\item
  Who (re)produces, maintains, and repairs what?
\item
  Who consumes what?
\end{itemize}

As the previous section has made clear, the answers depend on the
perspective we take and our understanding of the economy. Here we follow
Karl Polanyi, who proposes a broad understanding of the economy. Karl
Polanyi distinguishes between two definitions of economy: one formal and
one substantive (Polanyi (1977)). \textbf{From the formalist position},
economic activity is analyzed in the neoclassical sense as a
purpose-rational and utility-maximizing activity in markets under
conditions of scarcity. Since economic activity is reduced to market
economies, essential economic sectors such as domestic and care work or
public provisioning are only covered to a limited extent. Fundamental,
non-market institutions of economic activity are examined, but only from
the perspective of how these areas of human life can also be optimized.
However, this is problematic: rationality optimizes but does not concern
itself with what is being optimized, why, and for whom.

\textbf{From the substantivist position}, economic activity is
understood as being embedded in communities and biophysical foundations.
Economic activity therefore means providing the foundations of life. It
does not mean optimizing to maximize utility (Nelson (1993)). Markets
are not always suitable for securing the foundations of human life in
all economic sectors. Economics is therefore more than market economies.
The economy consists of different institutions and sectors with
different logics: housing cooperatives work with different business
models than plumbers and steel companies, public hospitals differ from
industrial companies, and unpaid care work is organized differently than
assembly line work. Karl Polanyi therefore distinguishes between four
socio-economic organizational principles or forms of provisioning (what
he calls forms of integration) that can be found in real economies and
interact in different ways: household, reciprocity, redistribution, and
market trade (Polanyi (2017)).

(1) \textbf{Householding} refers to forms of self-sufficiency and is
rooted in families and households. In ancient Greece, the oikos, or
household, was a self-sufficient economic unit. Even today, however, a
large part of the economy continues to take place in the household,
particularly in the form of unpaid care work, nursing, and housework
(e.g., cooking, cleaning, gardening).

(2) \textbf{Reciprocity} is based on the principle of give and take and
defines an exchange of goods and services between individuals outside
the market and the state. This takes place in communities, e.g., among
friends, in the neighborhood, or in associations, and includes
neighborhood assistance and community work. Like the principle of
householding, the form of provisioning/integration of reciprocity has
its roots often in local communities and usually takes place between
people who know each other.

(3) \textbf{Redistribution} defines a systematic flow of resources to an
administrative center and their subsequent redistribution. Examples
include public education, health, and pension systems financed by taxes
or levies. Redistribution allocates resources to (often unknown) members
of a society. It takes place within political territories, especially
the nation-state, and therefore extends beyond local communities.

(4) Finally, (\textbf{market) exchange} defines the exchange of goods
and services at market prices. This is the commodified area of economic
activity. These markets can differ depending on the type of goods and
services traded, their reach, and their structure. The logic of
individual profit and solvency prevails in market relationships.

In summary: To this day, economic activity is shaped by a variety of
institutions and principles. Real economies are always mixed economies,
i.e., they are more than market economies. Not all aspects of life and
economic activity are suitable for being transformed into goods that are
exchanged in a market (commodified).

\begin{tcolorbox}[enhanced jigsaw, breakable, colbacktitle=quarto-callout-tip-color!10!white, left=2mm, bottomtitle=1mm, colback=white, rightrule=.15mm, colframe=quarto-callout-tip-color-frame, bottomrule=.15mm, opacityback=0, titlerule=0mm, opacitybacktitle=0.6, toptitle=1mm, toprule=.15mm, title={Questions for reflection (.unnumbered)}, arc=.35mm, leftrule=.75mm, coltitle=black]

\begin{itemize}
\item
  Find further examples from your everyday life for all forms of
  provisioning.
\item
  Find an example of a good or service that is provided through all four
  forms of provisioning. How do the four forms of provisioning change
  the situation and character of the transaction/provision?
\end{itemize}

\end{tcolorbox}

\section{(Re-)production factors}\label{re-production-factors}

\textbf{Production factors} are all the factors that are necessary as
inputs for the production of an output, a good or a service. The
aggregate production factors of an economy are described as factor
endowment. The origins of production factors can be found in
18th-century physiocratic thinking, and they became established as a
concept in classical economics. According to classical economics, the
factors of production can be divided into three categories: land,
capital, and labor. Over time, this trinity has been repeatedly
criticized, expanded, and adapted. For example, entrepreneurship,
knowledge \& technology, and nature itself have been brought into play
as additional factors. However, these aspects are often subsumed under
one of the three original production factors.

In addition to neglecting other factors, this understanding is also
criticized for overemphasizing the sphere of production over
reproduction and maintenance, or for neglecting reproduction and
maintenance and only considering them when they are sold as services
(commodified). For example, a simple cup in your household is produced
only once but then has to be washed a thousand times. Infrastructure and
objects must be constantly repaired and maintained. And workers must
also be cared and provided for, i.e., reproduced.

\subsection{Land}\label{land}

The term ``land'' refers to \textbf{natural resources} used for
production. These include resources such as farmland, mineral resources,
and water. Companies must pay land rent (such as emission certificates
or concessions for hydroelectric power plants) for the use of land,
which is a scarce production factor. In some cases, natural capital is
still regarded as a ``pure input factor'' or ``only an input factor''
regardless of its comprehensive life-sustaining functions, and is taken
for granted. For example, a forest is reduced to its purely economically
exploitable dimension, such as the sale of wood and game. It is also
often neglected that nature is never available in the form of goods, but
must always be made usable and economically exploitable by humans
through work (Schaupp (2024)).

\subsection{Capital}\label{capital}

Companies acquire capital factors through \textbf{investment}. They do
this either by using surpluses from previous production periods or by
taking on loans. Capital factors include machinery, tools, buildings,
and other tangible capital. Monetary capital only becomes a factor of
production when it is converted into tangible capital.

\subsection{Labor}\label{labor}

Labor is the third factor of production and is necessary for raw
materials to be turned into goods and services. \textbf{Coordinating,
intellectual, and executive activities} are necessary for the targeted
use of capital in order to be able to produce. As already mentioned at
the beginning, care work is neglected here. This is especially true when
this work is unpaid. In classical and neoclassical economics, the focus
is strongly on \textbf{paid work,} while unpaid reproductive work is
taken for granted.

The analysis usually only records and considers statistics on paid work.
The \textbf{unemployment rate} is generally used as the most important
indicator. This is usually calculated as follows:

\[unemployment\,rate = \frac{number\,of\,registered\,unemployed}{labour\,force} \]

However, this indicator is only one of several indicators that are used.
The International Labour Organization (ILO) works with a different
notion of unemployment. From this perspective, all permanent residents
in Switzerland who are without work, are looking for work, and could
start work within a short period of time are considered
\href{http://www.bfs.admin.ch/bfs/de/home/statistiken/arbeit-erwerb/erhebungen/els-ilo.html}{\textbf{unemployed}}.
\href{https://www.seco.admin.ch/seco/de/home/seco/nsb-news.msg-id-103772.html}{In
2024, the unemployment rate in Switzerland averaged 2.4\%}, and the
\href{https://www.bfs.admin.ch/bfs/de/home/statistiken/arbeit-erwerb/erwerbslosigkeit-unterbeschaeftigung/erwerbslose-ilo.assetdetail.32370051.html}{seasonally
adjusted unemployment rate according to the ILO was 4.5\% in the fourth
quarter of 2024.} Unlike the official unemployment rate, the
unemployment rate according to the ILO also includes, for example,
self-employed persons, persons affected by long-term unemployment such
as those who have exhausted their unemployment benefits, and young
people who are not yet in employment immediately after completing their
education. Accordingly, the ILO unemployment rate is higher than the
official unemployment rate. However, such indicators say nothing about
the type of employment and working conditions, which means they are not
useful for such assessments. More fundamental questions about the
meaning and purpose of work and a high employment target cannot be
discussed using such indicators.

The following figure shows the development of the official unemployment
rate in Switzerland since 1920. Unemployment in Switzerland has always
been very low by international standards. During the post-war economic
boom in particular, there was almost no unemployment in Switzerland,
which is why the country attracted many workers from abroad -- who often
worked in precarious conditions. It was not until the crisis of the
1970s that unemployment returned to Switzerland, although the country
prevented a sharp rise by exporting unemployment abroad -- by 1977,
almost 250,000 foreign workers had to leave the country (Degen (2012),
p.~910).

\begin{figure}[H]

\centering{

\pandocbounded{\includegraphics[keepaspectratio]{images/Fig2_1_employementStat.png}}

}

\caption{\label{fig-employment}1920-1996 Statistical Yearbook of
Switzerland; 1997-2019 Seco}

\end{figure}%

Since 1997, the FSO (Federal Statistical Office) has been recording the
monetary value and \textbf{quantity of unpaid work in Switzerland} every
four years using the satellite account for household production
(\href{https://www.bfs.admin.ch/bfs/de/home/statistiken/arbeit-erwerb/erwerbstaetigkeit-arbeitszeit/vereinbarkeit-unbezahlte-arbeit/satellitenkonto-haushaltsproduktion.html}{SHHP}).
In Switzerland, 9.8 billion hours of unpaid work were performed in 2020
(mostly by women), more than was spent on paid work (7.6 billion hours).
The total value of unpaid work in 2020 is estimated at CHF 434 billion.
If GDP is expanded to include total household production, this
corresponds to 41.4\% of the value added of the expanded total economy.

\begin{tcolorbox}[enhanced jigsaw, breakable, colbacktitle=quarto-callout-tip-color!10!white, left=2mm, bottomtitle=1mm, colback=white, rightrule=.15mm, colframe=quarto-callout-tip-color-frame, bottomrule=.15mm, opacityback=0, titlerule=0mm, opacitybacktitle=0.6, toptitle=1mm, toprule=.15mm, title={Questions for reflection (.unnumbered)}, arc=.35mm, leftrule=.75mm, coltitle=black]

\begin{itemize}
\tightlist
\item
  Collect two examples for each of the three factors of production.
\item
  Collect examples of (re)production factors that are insufficiently
  represented in the three classic production factors.
\item
  Using examples, consider how the following constellations affect
  production:

  \begin{itemize}
  \tightlist
  \item
    if there is more labor than capital.
  \item
    if there is more capital than labor.
  \item
    if the factor land is not available.
  \item
    When all production factors are in balance.
  \end{itemize}
\end{itemize}

\end{tcolorbox}

\section{The marginal principle -- marginal utility and marginal
costs}\label{the-marginal-principle-marginal-utility-and-marginal-costs}

Before we look at the form of provisioning of market exchange, we will
focus on the marginal principle. This is central to the conception of
the market, as it provides an explanation for the formation of value and
utility and thus demand. The marginal principle has its origins in the
emergence of neoclassical economics around 1870 and still forms the
\textbf{basis of neoclassical economics} today. It deals with the
question of the value of goods, thereby also influencing the allocation
of goods, services, and factors of production. In contrast to classical
economics, the neoclassical marginal principle departs from the
objective theory of value and formulates the \textbf{subjective theory
of value}.

The \textbf{objective theory of value} in classical economics postulates
that all goods have an objective value. The labor invested in the
production of goods was commonly used as the basis for determining this
value, which gave rise to the labor theory of value. In
\textbf{subjective theory of value}, on the other hand, no absolute
values are considered, but rather the subjective values that people
ascribe to goods and their marginal changes in relation to the initial
situation. This focus on marginal changes forms the basis of the
marginal principle. The marginal principle enabled differential calculus
to be applied to cost/benefit and revenue functions, which led to the
sustained rise of mathematical models in economics.

\subsection{Revenue function of land and marginal
revenue}\label{revenue-function-of-land-and-marginal-revenue}

The marginal principle was originally observed in the revenue function
of land. Studies of agriculture in the 18th century revealed the
relationship between factor input and revenue (originally yield in
agricultural production). \textbf{Marginal revenue} is the increase in
yield resulting from an additional unit of a production factor.

\begin{quote}
``It is in the nature of agriculture -- and this is a very noteworthy
circumstance -- that the increase in production does not rise in direct
proportion to the number of workers employed, but that each worker
employed later produces less than the previous one.'' - Johann Heinrich
von Thünen, Der isolirte Staat, 1850. Quoted in Winfried Reiss,
Mikroökonomische Theorie: Historisch fundierte Einführung, 2007,p.90,
own translation
\end{quote}

What von Thünen observed with the factor of labor was also noted by
A.R.J. Turgot with the use of the factor of capital in the form of
fertilizer. The principle of marginal productivity can also be applied
to industrial production and, for example, to the number of workers in a
factory.

\subsection{Gossen's laws}\label{gossens-laws}

In 1854, Hermann Heinrich Gossen developed his \textbf{laws of human
interatctions} and the rules for human behavior that follow from them.
In them, he formulated the regularity of individual preferences and
their satisfaction, which is based on the respective utility.

\subsubsection{Gossen's first law: the Law of Diminishing Marginal
Utility}\label{gossens-first-law-the-law-of-diminishing-marginal-utility}

The law of diminishing marginal utility, also known as \textbf{the law
of diminishing marginal utility}, states:

\begin{quote}
``The magnitude of one and the same pleasure diminishes continuously as
we continue to enjoy it, until finally saturation is reached.''
\end{quote}

\begin{quote}
Hermann Heinrich Gossen, Entwickelung der Gesetze des menschlichen
Verkehrs, und der daraus fliessenden Regeln für menschliches Handeln,
Braunschweig 1854,p.4, own translation.
\end{quote}

Thus, the utility of the first cup of coffee in the morning is greater
than that of the third or fourth. The same applies to living space,
where, for example, the first bathroom in an apartment is of greater
utili-ty to the residents than the second. According to the theory, this
law can be applied to all goods.

\subsubsection{Gossen's second law: the Law of Equi-marginal
Utility}\label{gossens-second-law-the-law-of-equi-marginal-utility}

The law of equal marginal utility, also known as the law of
equi-mariginal utility, states:

\begin{quote}
``A person who is free to choose between several pleasures but does not
have enough time to enjoy them all to the fullest, must, however
different the absolute magnitude of the individual pleasures may be, in
order to maximize the sum of his pleasure, before he fully prepares even
the greatest of them, prepare them all partially, and in such a
proportion that the magnitude of each pleasure remains the same for all
at the moment when its preparation is interrupted.''
\end{quote}

\begin{quote}
Hermann Heinrich Gossen in Entwicklung der Gesetze des menschlichen
Verkehrs, und der daraus fliessenden Regeln für menschliches Handeln
1854, p.12, own translation
\end{quote}

This law aims at the optimal use of limited resources for various goods.
The optimal utility for an individual is achieved when the marginal
utility (i.e., the utility of the next additional unit) is the same for
all goods. Mathematically, this means that the marginal utility
functions of the various goods are equated. The marginal utility
corresponds to the maximum price that the person in question is willing
to pay. The aggregate demand (function) can be determined by aggregating
the marginal utility (functions) of all individuals. The marginal
principle thus forms the core of neoclassical consumption theory. For
interested students (who are familiar with the basics of neoclassical
economics), we recommend a critical discussion of neoclassical
consumption theory by Ben Fine.

Marginal costs and marginal revenue in companies As derived above,
marginal revenue describes the revenue from an additional unit.
Similarly, the term marginal cost describes the cost of an additional
unit. Marginal costs can be determined by deriving the mathematical cost
function. The costs {[}C{]} of a company are the sum of the fixed costs
{[}C\textsubscript{f}{]} (e.g., rent, machinery) and the variable costs
{[}C\textsubscript{v}{]} (labor, raw materials, intermediate products).

\[ C = C_f+C_V \]

According to neoclassical theory, companies are guided not only by the
cost function but also by the revenue function. This describes how much
revenue {[}R{]} is generated from the sale of a good depend-ing on
quantity {[}Q{]} and price {[}P{]}. Deriving the revenue function yields
the marginal revenue, which is the revenue from an additional unit sold.

\[ R = Q * P \]

A company produces until the marginal costs exceed the marginal
revenues. This can be calculated by equating the derivatives of the cost
and revenue functions. In market equilibrium (under perfect
com-petition), theoretically the price corresponds to the marginal
costs. Although neoclassical theory focuses heavily on the price, it has
paid little attention to how companies set prices. In his dissertation,
Nubbemeyer (2010) shows that until the middle of the 20th century, there
was a controversial debate about whether companies actually base their
pricing on the marginal principle or on alternative concepts (e.g.,
mark-up pricing as in some post-Keynesian models). Although the debate
was not conclusively resolved, the marginal principle prevailed with the
advent of neoclassical economics.

\begin{tcolorbox}[enhanced jigsaw, breakable, colbacktitle=quarto-callout-tip-color!10!white, left=2mm, bottomtitle=1mm, colback=white, rightrule=.15mm, colframe=quarto-callout-tip-color-frame, bottomrule=.15mm, opacityback=0, titlerule=0mm, opacitybacktitle=0.6, toptitle=1mm, toprule=.15mm, title={Questions for reflection (.unnumbered)}, arc=.35mm, leftrule=.75mm, coltitle=black]

\begin{itemize}
\tightlist
\item
  How many employees does a company hire according to the marginal
  principle?
\item
  According to theory, what determines the level of employees' wages?
\item
  What does this mean for involuntary unemployment? When does it occur?
\item
  Could Gossen's first law be used to argue for income redistribution?
\end{itemize}

\end{tcolorbox}

\section{How the market works}\label{how-the-market-works}

As described above, in neoclassical economics, i.e., mainstream
economics, the \textbf{market} is regarded \textbf{as the central
institution} for the provision of goods and services. We will therefore
take a closer look at how the market is conceptualized in neoclassical
economics.

\subsection{The market as a central coordinating
authority}\label{the-market-as-a-central-coordinating-authority}

In markets, supply and demand meet and serve as \textbf{the basis for
the price mechanism}. Supply is the willingness of an actor to offer
goods, services, or factors, and demand is the willingness to purchase
them. The utility functions of the actors, which can be calculated on
the basis of the marginal principle, form the basis for demand.
Aggregate demand is the sum of the individual utility functions.

In economics, markets are usually represented by price-quantity
diagrams. The axes represent quantity and price (or wages in labor
markets and interest rates in capital markets), while supply and demand
are represented by curves. The slope of the curves changes with the
elasticity of supply and demand and shows how strongly they change when
quantity or price changes.

\begin{figure}[H]

\centering{

\pandocbounded{\includegraphics[keepaspectratio]{images/Fig2_2_SupplyDemand.png}}

}

\caption{\label{fig-supplyDemand}Graphical representation and
determination of a market equilibrium}

\end{figure}%

If the supply of a good is too high, the price falls; if demand is too
high, the price rises. In market equilibrium, demand equals supply and
the price stabilizes. The price mechanism thus determines who produces
what, how much is produced, and for whom. The price is therefore
regarded as the central allocation instrument in markets. For example, a
factory that produces with outdated machinery cannot produce at the same
price as another factory with more modern equipment. Due to the high
production costs, it is forced out of the market and must produce other
goods or file for bankruptcy. A central assumption for the supply curve
to have an upward slope is that above a certain size the marginal costs
of companies rise again (even in the long term), for example due to
increased coordination costs, meaning that there is a limit to growth
for companies. However, this assumption is controversial (see, for
example, Keen (2010), Pirgmaier (2017)). If this assumption is not
fulfilled, the supply curve could also have a different slope and there
would be no guaranteed market equilibrium with many competitive
companies, but rather a tendency toward monopoly. Even Alfred Marshall,
the founder of this price-quantity diagram, repeatedly struggled with
the question of whether there could also be a tendency toward monopoly
(Marshall (1919), p.~316; Marshall (1919), p.~324).

\begin{tcolorbox}[enhanced jigsaw, breakable, colbacktitle=quarto-callout-note-color!10!white, left=2mm, bottomtitle=1mm, colback=white, rightrule=.15mm, colframe=quarto-callout-note-color-frame, bottomrule=.15mm, opacityback=0, titlerule=0mm, opacitybacktitle=0.6, toptitle=1mm, toprule=.15mm, title={Brief digression: Price level)}, arc=.35mm, leftrule=.75mm, coltitle=black]

Given the importance of the price mechanism for the functioning of
markets, the price level in an economy is one of the most important
variables to be observed and analyzed. Price stability is measured by
the inflation rate (rate of price increases) of goods and services. Low
inflation of up to 2\% is considered a stable price level. Statistical
offices use a standardized ``basket of goods'' to record how the price
index of the cost of living changes over time. If the price level falls,
this is referred to as deflation. The composition of the basket of goods
varies from country to country and usually covers only markets for
goods, but not capital markets.

In Switzerland, the Federal Statistical Office (FSO) compiles the
National Consumer Price Index (CPI). The figure below shows the consumer
goods that make up this index and their respective shares. The basket of
goods contains ``the most important goods and services consumed by
private households.'' The weighting is based on annual household surveys
and is intended to reflect actual consumption patterns as accurately as
possible.

\begin{figure}[H]

\centering{

\pandocbounded{\includegraphics[keepaspectratio]{images/Fig2_3_ConsBasket.png}}

}

\caption{\label{fig-ConsBasket}Composition of the basket of goods for
the national consumer price index 2023 (Federal Statistical Office,
2023)}

\end{figure}%

\end{tcolorbox}

\subsubsection{The invisible hand of the
market}\label{the-invisible-hand-of-the-market}

The term ``invisible hand'' is often used to describe the
\textbf{self-regulation of markets}. The term was coined by Adam Smith.
Adam Smith used the term twice in his earlier work ``The Theory of Moral
Sentiments'' (1759) and once in his work ``The Wealth of Nations''
(1776). In ``The Wealth of Nations,'' Smith uses the term in the chapter
on ``Of Restraints upon the Importation from Foreign Countries of such
Goods as can be Produced at Home''. In this context, Smith writes that
if someone prefers to support the national economy instead of the
foreign economy, that person is guided by an invisible hand. Noam
Chomsky therefore interprets Smith as describing a ``home bias'' in this
passage. However, various economists have interpreted the passage
differently. The most popular interpretation sees this passage as
establishing the self-regulating forces of the market. Market forces are
a regulating force that leads individuals to satisfy their individual
needs in the best possible way, which at the same time serves society by
distributing goods in the best possible way. Such an interpretation is
controversial in view of the surrounding context and is repeatedly
criticized by experts in the history of economic thought.

\begin{quote}
``As every individual, therefore, endeavours as much as he can both to
employ his capital in the support of domestic industry, and so to direct
that industry that its produce may be of the greatest value; every
individual necessarily labours to render the annual revenue of the
society as great as he can. He gener-ally, indeed, neither intends to
promote the public interest, nor knows how much he is promoting it. and
he is in this, as in many other cases, led by an invisible hand to
promote an end which was no part of his intention. Nor is it always the
worse for the society that it was no part of it. By pursuing his own
interest he frequently promotes that of the society more effectually
than when he really intends to promote it. I have never known much good
done by those who affected to trade for the public good.''
\end{quote}

\begin{quote}
Smith, Adam. An Inquiry into the Nature and Causes of the Wealth of
Nations. Edited by Edwin Cannan. Chicago: University of Chicago Press,
1976.
\end{quote}

\subsubsection{Perfect competition}\label{perfect-competition}

A basic prerequisite for the functioning of markets is that of perfect
competition. In \textbf{perfect competition}, individual suppliers have
no influence on the price, because if the price increases, consumers
will buy from other suppliers. Companies must accept the market price as
a given and can only decide on the quantity they offer. They are
therefore in the role of price takers and quantity adjusters.

For a market to be considered perfectly competitive, several conditions
must be met:

\begin{itemize}
\item
  \textbf{Large mumber of Market participants}: There must be many
  \textbf{suppliers} and \textbf{consumers}. This structure is referred
  to as a \emph{\textbf{polypoly} (in German ``Polypol'', however, in
  English usually just referred to as perfect competition)}. In
  contrast, a market with only a few suppliers or consumers is called an
  \textbf{\emph{oligopoly}} (few suppliers) or
  \textbf{\emph{oligopsony}} (few consumers). If there is only one
  supplier or consumer, the market is a \textbf{\emph{monopoly}} or
  \textbf{\emph{monopsony}}, respectively (see diagram below for an
  overview).
\item
  \textbf{Perfect information}: All participants must have full and
  equal access to relevant information. There should be no hidden
  agreements between market actors or undisclosed quality deficiencies
  in the goods offered.
\item
  \textbf{No consumer preferences for specific suppliers}: Consumers
  must be indifferent to which supplier they buy from. A perfectly
  competitive market requires that buyers can switch freely between
  suppliers. If consumers favor a specific supplier, that supplier could
  charge a price above the market rate without losing demand, which
  violates the condition of perfect competition.
\end{itemize}

If the above conditions are not met, there is a \textbf{distortion of
competition}. In a neoclassical market economy, the state should
intervene with regulations in such a situation.

\begin{longtable}[]{@{}
  >{\raggedright\arraybackslash}p{(\linewidth - 6\tabcolsep) * \real{0.2717}}
  >{\raggedright\arraybackslash}p{(\linewidth - 6\tabcolsep) * \real{0.2500}}
  >{\raggedright\arraybackslash}p{(\linewidth - 6\tabcolsep) * \real{0.2826}}
  >{\raggedright\arraybackslash}p{(\linewidth - 6\tabcolsep) * \real{0.1739}}@{}}
\caption{Overview of the different market forms. Polypoly is a condition
for the assumption of perfect
competition}\label{tbl-marketforms}\tabularnewline
\toprule\noalign{}
\begin{minipage}[b]{\linewidth}\raggedright
\end{minipage} & \begin{minipage}[b]{\linewidth}\raggedright
one buyer
\end{minipage} & \begin{minipage}[b]{\linewidth}\raggedright
few buyers
\end{minipage} & \begin{minipage}[b]{\linewidth}\raggedright
many buyers
\end{minipage} \\
\midrule\noalign{}
\endfirsthead
\toprule\noalign{}
\begin{minipage}[b]{\linewidth}\raggedright
\end{minipage} & \begin{minipage}[b]{\linewidth}\raggedright
one buyer
\end{minipage} & \begin{minipage}[b]{\linewidth}\raggedright
few buyers
\end{minipage} & \begin{minipage}[b]{\linewidth}\raggedright
many buyers
\end{minipage} \\
\midrule\noalign{}
\endhead
\bottomrule\noalign{}
\endlastfoot
One seller/supplier & Bilateral Monopoly & Limited Supply Monopoly &
Monopoly \\
Few sellers/suppliers & Restricted Monopsony & Bilateral Oligopoly &
Oligopoly \\
Many sellers/suppliers & Monopsony & Oligopsony & Polypoly \\
\end{longtable}

\begin{figure}[H]

\centering{

\pandocbounded{\includegraphics[keepaspectratio]{images/Fig2_4_TypeMarkets.png}}

}

\caption{\label{fig-TypeMarkets}Different types of markets}

\end{figure}%

This market mechanism for coordinating supply and demand is applied to
various markets. A fundamental distinction is made between goods markets
and factor markets. Goods and services are traded on \textbf{goods
markets}. On \textbf{factor markets,} the factors of production (land,
capital, labor) are supplied and demanded. For example, households offer
their labor on the labor market, and instead of a price, the demand from
companies determines the level of wages paid through the price
mechanism. Karl Polanyi describes how the introduction of factor markets
led to the development of today's ``market society'' and how social
processes are increasingly aligning themselves with market logic
(Polanyi (2017)).

\begin{tcolorbox}[enhanced jigsaw, breakable, colbacktitle=quarto-callout-tip-color!10!white, left=2mm, bottomtitle=1mm, colback=white, rightrule=.15mm, colframe=quarto-callout-tip-color-frame, bottomrule=.15mm, opacityback=0, titlerule=0mm, opacitybacktitle=0.6, toptitle=1mm, toprule=.15mm, title={Questions for reflection (.unnumbered)}, arc=.35mm, leftrule=.75mm, coltitle=black]

\begin{itemize}
\tightlist
\item
  For all three conditions of perfect competition, think of examples
  from today's economy where these conditions are met and where they are
  not.
\item
  Are there differences in the demand for different goods? How does this
  affect markets for food, cars, and leisure activities?
\item
  Look up the inflation rate on the website of the Swiss Federal
  Statistical Office (bfs.admin.ch) and describe its development in
  Switzerland since 2000.
\end{itemize}

\end{tcolorbox}

\section{Economic actors}\label{economic-actors}

Economic activity is shaped by various actors. They all have different
roles, interests, and possibilities for action. Such actors include
companies, households, the state, civil society, financial institutions,
etc. Different degrees of complexity or perspectives mean that more or
fewer of these actors are included in the analysis. And depending on how
the economy is conceptualized, the roles of the actors and how they
interact with each other also change. The circular flow analysis is the
most common concept of economic activity, but there are many different
variants.

\subsection{Circular flow analysis}\label{circular-flow-analysis}

In the circular flow analysis, which dates back to François Quesnay
(1694-1774), economic activities are recorded as flows of goods and
money between different actors and represented graphically. The main
actors are households and companies, and in more complex
representations, the state, banks, and foreign countries are also
included. The graphical representation of economic processes serves to
simplify the modeling of theoretical assumptions.

In a simple circular flow of the economy, households and companies are
linked. The former provide the latter with their labor in exchange for
wages. Using labor and other factors of production, companies produce
goods and services that households consume.

In a circular flow of the economy, expenditure always equals income at
an aggregate level. Households receive wages (\emph{W}) and can consume
these again as expenditure (\emph{C}). Companies receive profits
(\emph{P}) and spend these again in the form of investment (\emph{I}).

\[ Revenue = Expenditure \] \[ W+P = C+I \]

\pandocbounded{\includegraphics[keepaspectratio]{images/Fig2_5_economicCycle.png}}
This simple circular flow can be extended to illustrate more complex
relationships by adding other actors.

\pandocbounded{\includegraphics[keepaspectratio]{images/Fig2_6_economicCycleExt.png}}By
describing economic relationships, circular flow analysis makes it
possible to explain and predict phenomena. It also provides a basis for
economic policy measures based on systematic investigations of economic
relationships, connections, and regularities.

In most conventional circular flow analyses, only the market economy is
represented as a producing and value-creating entity. Households are
merely portrayed as consumers. Unpaid work and (re)production that take
place outside the sphere of the market economy are thus systematically
neglected and devalued. Feminist economics has formulated this criticism
and has dealt extensively with alternative concepts. Diane Elson's
circular flow model serves as an illustration here. If you are
interested, you can explore the perspective of feminist economics in
more depth in section 4 on pluralist economics.

\begin{figure}[H]

\centering{

\pandocbounded{\includegraphics[keepaspectratio]{images/Fig2_7_CircularFlow.png}}

}

\caption{\label{fig-CircularFlow3}Circular flow model according to Diane
Elson 2000}

\end{figure}%

\begin{tcolorbox}[enhanced jigsaw, breakable, colbacktitle=quarto-callout-tip-color!10!white, left=2mm, bottomtitle=1mm, colback=white, rightrule=.15mm, colframe=quarto-callout-tip-color-frame, bottomrule=.15mm, opacityback=0, titlerule=0mm, opacitybacktitle=0.6, toptitle=1mm, toprule=.15mm, title={Questions for reflection (.unnumbered)}, arc=.35mm, leftrule=.75mm, coltitle=black]

\begin{itemize}
\tightlist
\item
  What happens in the simple circular flow model when total economic
  expenditure remains the same and corporate profits rise at the same
  time?
\item
  How does falling consumer spending affect the economy in this model?
\item
  How does Elson's circular model differ from the two models shown
  above? What does this mean for the respective understanding of the
  economy?
\end{itemize}

\end{tcolorbox}

\section{Division of labor}\label{division-of-labor}

Alongside the invisible hand, Adam Smith's \textbf{needle factory} is
probably the best-known excerpt from his work. The first chapter deals
with the division of labor in a needle factory and how this greatly
increases productivity. Based on this, Adam Smith makes a more general
observation \textbf{about how the division of labor increases
productivity}. Individuals, companies, or countries that specialize in a
particular activity can perform it increasingly better. This gives them
an absolute advantage in production over others. Economic entities
increasingly specialize in those areas in which they are better than
others. If an entity is not superior to another in any area, it cannot
participate in activities around market exchange.

David Ricardo later criticizes Smith's theory of absolute advantages.
According to Smith, if one economic entity were superior to another in
all respects, it would produce everything itself. Ricardo, on the other
hand, established the \textbf{theory of comparative advantage.} This
states that entities produce those goods in which they have the greatest
relative advantage or the least relative disadvantage compared to
others. In this case, it still makes sense for entities that are not
superior to others in any activity to participate in a common economy.
In doing so, they take on tasks for which others are better qualified
but do not prioritize due to a lack of resources.

\textbf{Example}: Ricardo illustrated his theory using the example of
trade (cloth and wine) between England and Portugal, aiming to show that
free trade would be beneficial for both sides. It was a plea against the
import restrictions and tariffs between countries that were prevalent at
the time and based on mercantilist logic. This short video illustrates
the basic idea behind Ricardo's theory of comparative cost advantages:

\url{https://www.youtube.com/watch?v=0hK9p8BSDMM}

Ricardo is often cited to illustrate that free trade is beneficial for
all countries involved. Although widespread, this perspective should be
viewed critically. For one thing, the term ``free trade'' quickly loses
its meaning in the context of European imperialism, which was emerging
at the time Ricardo developed his theory. During this period, the global
market was shaped by the prevailing power relations, primarily to the
advantage of the Global North. It was based on the exploitation of the
Global South and created unequal structures, some of which persist to
this day (see, for example, Anievas and Nişancioğlu (2015), M. Davis
(2017), Ghosh (2021), Hickel (2017)). In addition, advantageous
participation in global trade requires a country to have competitive
industries. Western Europe and the US also built up their industries
under the protection from the global market and with industrial policies
(Chang (2002)). The forced economic opening of many countries in the
Global South from the end of the 1970s onwards by institutions such as
the International Monetary Fund made it difficult for them to develop
their own competitive manufacturing industries and pushed them back into
the production or extraction of raw materials (see, for example, Palma
(2003)). Dorninger et al. (2021) show how strongly this unequal exchange
manifests and materializes in various aspects (e.g., land use, resource
consumption, etc.).

\subsection{Types of division of
labor}\label{types-of-division-of-labor}

From an economic perspective, division of labor can be divided into
three main types: intra-company division of labor, inter-company
division of labor, and international division of labor.

\textbf{Intra-company division of labor} describes the division of tasks
within a company that was prevalent in the course of industrialization.
Mass production in factories replaced serial production in
manufactories. According to Taylorist (after Frederick W. Taylor,
1856-1915) and Fordist (after Henry Ford) principles, work was divided
into small, repetitive steps. Workers were no longer responsible for
manufacturing an entire product, but only for a small step along the
assembly line.

In the \textbf{inter-company division of labor}, production processes
are distributed across economic actors. An example of this is the spread
of supplier companies, which can produce intermediate products (e.g.,
semiconductors) more cheaply for other companies due to their
specialization.

Finally, the \textbf{international division of labor} describes the
division of labor at the global level. With the increase in
globalization since the 1980s, the interdependence of the global economy
has steadily grown. In addition, the division of labor is increasingly
taking place along the value chains of individual goods and not just
between goods. A separate field of research has developed on such global
value chains.

\begin{tcolorbox}[enhanced jigsaw, breakable, colbacktitle=quarto-callout-tip-color!10!white, left=2mm, bottomtitle=1mm, colback=white, rightrule=.15mm, colframe=quarto-callout-tip-color-frame, bottomrule=.15mm, opacityback=0, titlerule=0mm, opacitybacktitle=0.6, toptitle=1mm, toprule=.15mm, title={Questions for reflection (.unnumbered)}, arc=.35mm, leftrule=.75mm, coltitle=black]

\begin{itemize}
\tightlist
\item
  Consider two advantages and two disadvantages of the division of labor
  at the company, inter-company, and international levels.
\item
  Do you know of any specific examples of national specialization?
  Research how this specialization was implemented.
\item
  What could be the disadvantages or risks of international division of
  labor?
\end{itemize}

\end{tcolorbox}

\section{Categories of goods}\label{categories-of-goods}

Goods are all things and services that have been made usable from nature
or produced by humans to satisfy needs. Their various characteristics
and the social structures surrounding them significantly influence how
different actors interact with them and how their use is regulated.

\subsection{Good characteristics}\label{good-characteristics}

We categorize goods depending on whether or not people can be prevented
from consuming them (\textbf{excludability}) and whether or not they can
consume them without affecting their availability to other individuals
(\textbf{rivalry}).

All goods can be divided into one of \textbf{four types} based on these
two characteristics of \textbf{rivalry} and \textbf{exclusivity}.

\begin{longtable}[]{@{}
  >{\raggedright\arraybackslash}p{(\linewidth - 4\tabcolsep) * \real{0.1678}}
  >{\raggedright\arraybackslash}p{(\linewidth - 4\tabcolsep) * \real{0.4765}}
  >{\raggedright\arraybackslash}p{(\linewidth - 4\tabcolsep) * \real{0.3490}}@{}}
\caption{Overview of types of goods based on the characteristics of
rivalry and exclusivity.}\label{tbl-typeofgoods}\tabularnewline
\toprule\noalign{}
\begin{minipage}[b]{\linewidth}\raggedright
Characteristics
\end{minipage} & \begin{minipage}[b]{\linewidth}\raggedright
rival in use
\end{minipage} & \begin{minipage}[b]{\linewidth}\raggedright
non-rival in use
\end{minipage} \\
\midrule\noalign{}
\endfirsthead
\toprule\noalign{}
\begin{minipage}[b]{\linewidth}\raggedright
Characteristics
\end{minipage} & \begin{minipage}[b]{\linewidth}\raggedright
rival in use
\end{minipage} & \begin{minipage}[b]{\linewidth}\raggedright
non-rival in use
\end{minipage} \\
\midrule\noalign{}
\endhead
\bottomrule\noalign{}
\endlastfoot
\textbf{excludable use} & \ul{\textbf{Private goods}}

bread apartment clothing & \ul{\textbf{Club goods}}

cable television streaming services national park \\
\textbf{non-excludable use} & \ul{\textbf{Common goods}}

high seas fishing grounds Gotthard road tunnel on the Easter weekend &
\ul{\textbf{Public goods}}

legal system waterway corrections fireworks \\
\end{longtable}

\textbf{Private goods} are both excludable and rival in consumption.
This means that it is possible to prevent people from consuming them
through measures and that the consumption of a good cannot be repeated
by any number of people. Because of these characteristics, private goods
are characterized as good for market exchanges.

\textbf{Club goods} are excludable but non-rival in consumption. For
example, cable television or streaming platforms can exclude people from
consumption in the form of a subscription fee, but do not suffer if the
number of users increases. Outside the realm of digital goods, the
distinction between private and club goods is more difficult to make.
For example, membership in a fitness club and use of the associated
facilities is only non-rival until January, when all users want to
pursue their New Year's resolutions and all equipment is fully utilized.

\textbf{Common goods} are non-excludable but rival in consumption. These
often include natural resources such as fish stocks or the availability
of water for irrigating fields. Coordination between the various users
is essential to prevent overuse of these goods.

\textbf{Public goods} are neither excludable nor rival in consumption.
Such goods are hardly tradable on markets due to their
non-excludability. The use of public goods without paying for them is
described as the free rider problem. However, such necessary goods can
still be provided through government funding.

However, the classification of these goods is not a natural one.
Excludability in particular is not an inherent characteristic of a good,
but depends primarily on technical possibilities, social negotiation,
and social power relations. When dealing with natural resources in
particular, the question of exclusivity is primarily a social and/or
technical one. Water resources, for example, can be made technically
accessible without any problems, but they can also be made exclusive.
Even sunlight could theoretically be made exclusive, as Mr.~Burns plans
to do in the TV series ``The Simpsons.''
\href{https://www.youtube.com/watch?v=L3LbxDZRgA4}{From his point of
view, the sun is the enemy.}

When categorizing goods, it is also important to note that this involves
a process of reification. Goods are reduced to their purely economic
value. This can be very problematic, especially for resources that are
managed as common goods (also known as commons). It often obscures how
commons structure social relationships and communities, enable
participation, and maintain a relationship with nature and the
environment. Ugo Mattei therefore argues that it is simplistic to say
that we ``have'' a common good. Rather, it is a question of exploring
the extent to which we ``are'' commons, insofar as we are also part of
the environment (Mattei 2014, p.~76). For this reason, social
anthropology, for example, often uses the verb ``commoning'' instead of
the noun ``commons.''

\textbf{The tragedy of the commons}

Actors cannot be excluded from consuming public and common goods. This
can result in overuse of these goods, which was described by Garret
Hardin in 1968 as the tragedy of the commons. Privatization of these
goods (known as enclosure) is often proposed as a solution; whereby
clear property rights can be assigned to exclude actors from use.\\
\strut \\
Elinor Ostrom researched the same topic and observed how self-organized
cooperation within clear boundaries can counteract overuse even without
the allocation of property rights. However, effective solutions to the
overuse of non-excludable goods are highly context-dependent and
therefore difficult to generalize. Ostrom was awarded the Alfred Nobel
Memorial Prize in Economic Sciences in 2009 for her research on the
commons. The now extensive research on the commons includes many
examples of commons that have not been overused and have not succumbed
to tragedy. There are examples of commons that have been managed
collectively and sustainably for centuries (see, for example, De Moor
(2015); De Keyzer (2018)).\\
\strut \\
The commons research conducted by Ostrom and others shows that there are
solutions beyond privatization and nationalization to prevent the
tragedy of the commons. This also fundamentally breaks down the
dichotomy between the market and the state and enables points of contact
beyond these two spheres.

\subsection{External effects}\label{external-effects}

The price paid for the use of goods does not always correspond to their
true social cost. If the good is used at such a price anyway, the costs
are passed on to third parties, i.e., neither to producers nor
consumers, but to taxpayers, future generations, or nature, for example.
When this happens, we speak of \textbf{external effects}. External
effects can have a positive or negative impact on the production or
consumption possibilities of third parties. In neoclassical economics,
external costs are generally understood as an exception to an otherwise
well-functioning market. In contrast, William Kapp pointed out as early
as 1950 that the more a system is based on maximizing private profit,
the greater the incentives to pass on costs to people, society, and
nature (W. Kapp (1950)). Accordingly, the current capitalist system has
inherent tendencies toward the externalization of costs. Anna Saave
recently theorized this further (Saave (2022)). Stephan Lessenich
describes affluent societies in the Global North as externalization
societies that shift costs to the Global South (Lessenich (2020)).

\begin{longtable}[]{@{}
  >{\raggedright\arraybackslash}p{(\linewidth - 4\tabcolsep) * \real{0.1009}}
  >{\raggedright\arraybackslash}p{(\linewidth - 4\tabcolsep) * \real{0.3899}}
  >{\raggedright\arraybackslash}p{(\linewidth - 4\tabcolsep) * \real{0.5046}}@{}}
\caption{Overview of types of external
effects.}\label{tbl-exteffects}\tabularnewline
\toprule\noalign{}
\begin{minipage}[b]{\linewidth}\raggedright
\end{minipage} & \begin{minipage}[b]{\linewidth}\raggedright
Effect on\ldots{}

\ldots production possibilities of third parties
\end{minipage} & \begin{minipage}[b]{\linewidth}\raggedright
Effect on\ldots{}

\ldots consumption possibilities of third parties
\end{minipage} \\
\midrule\noalign{}
\endfirsthead
\toprule\noalign{}
\begin{minipage}[b]{\linewidth}\raggedright
\end{minipage} & \begin{minipage}[b]{\linewidth}\raggedright
Effect on\ldots{}

\ldots production possibilities of third parties
\end{minipage} & \begin{minipage}[b]{\linewidth}\raggedright
Effect on\ldots{}

\ldots consumption possibilities of third parties
\end{minipage} \\
\midrule\noalign{}
\endhead
\bottomrule\noalign{}
\endlastfoot
\textbf{positive effect} & Apple growers and beekeepers benefit mutually
from pollination and apple blossoms. & Parents have their children
vaccinated -\textgreater{} society benefits from lower infection rates
and healthcare costs \\
\textbf{negative effect} & Industry emits pollutants into waterways
-\textgreater{} third parties can no longer fish & Industry emits
pollutants into water -\textgreater{} third parties can no longer swim
in the river \\
\end{longtable}

\begin{tcolorbox}[enhanced jigsaw, breakable, colbacktitle=quarto-callout-tip-color!10!white, left=2mm, bottomtitle=1mm, colback=white, rightrule=.15mm, colframe=quarto-callout-tip-color-frame, bottomrule=.15mm, opacityback=0, titlerule=0mm, opacitybacktitle=0.6, toptitle=1mm, toprule=.15mm, title={Questions for reflection (.unnumbered)}, arc=.35mm, leftrule=.75mm, coltitle=black]

\begin{itemize}
\tightlist
\item
  Collect two examples of each of the four categories of goods that you
  encounter in everyday life.
\item
  What are the advantages and disadvantages of privatizing common goods?
  Consider this using a specific example.
\item
  Research two examples of external effects and how they have been/are
  being addressed politically.
\end{itemize}

\end{tcolorbox}

\section{Costs}\label{costs}

\subsection{Cost types}\label{cost-types}

If a company wants to provide a service, it has to pay its employees a
wage. If a student goes out for a beer with friends after class, she
cannot review the material she learned instead. And if farmers use too
much fertilizer on their fields, the quality of the surrounding water
will decline. Although different in different ways, in all of these
scenarios costs occur. Costs are negative consequences that arise when
creating value.

\textbf{Production costs} arise in the production of goods and services
in companies when production factors are consumed for their creation. A
distinction is often made between constant capital (machinery, tools)
and variable capital (labor).

\textbf{Marginal costs} describe the increase in costs caused by an
additional unit produced. As described above, this parameter is used in
combination with marginal revenue to assess the quantity to be produced.

\textbf{Opportunity costs} are the costs of a lost utility that could
have been obtained instead of the chosen use of the production factor or
good. In the example above, the student decided to invest her time in
social contacts instead of repeating the material. In this case, the
opportunity costs is the utility that was not gained by repeating the
material during this time. Since people always have different options
for using their time, costs always arise according to the concept of
opportunity costs. The famous saying \emph{``there is no such thing as a
free lunch''} means exactly that.

\textbf{Transaction costs} are the costs incurred in the process of
transactions. For example, when purchasing a piece of land, a buyer pays
a notary for certification and other costs, such as a land registry
entry, in addition to the price of the land.

\textbf{External costs} arise when costs are not passed on to those who
cause them (producers/consumers) but to third parties (society). Because
these costs are not included in the true price of goods, they create
incentives to consume the goods concerned in excessive quantities.
External costs often take the form of environmental damage, as the value
of nature is difficult to quantify.

\subsection{Cost-benefit analysis}\label{cost-benefit-analysis}

Cost-benefit analysis makes it possible to weigh up the costs and
benefits. This tool allows the expected consequences of decisions to be
assessed by weighing up the associated costs and benefits. The important
prerequisite for this analysis is the monetary valuation of all
consequences. Only in this way can an assessment be made using
comparable variables. The establishment of cost-benefit analysis dates
back to the discussion of social costs (external costs) in the 1950s and
1960s. Based on the work of economists such as William W. Kapp (1950),
it was argued that the resulting social costs should be prevented
through regulation. Ronald Coase (1960), on the other hand, argued that
regulations also represent costs for companies -- opportunity costs due
to lost profits. For example, the profit of a chemical company that
pollutes a river through its activities. This profit must be offset
against the social costs. According to Coase, such a total calculation
provides a basis for arguing against regulation and for the expansion of
property rights (i.e., the market).

The cost-benefit analysis has several problems. One difficulty is that
the value of many areas of social and natural life cannot be clearly
quantified with a price. For example, it is not possible to put a clear
and objective price on an endangered animal or plant species, the
existence of a forest, clean air, or a human life.

To meet this challenge, researchers have developed various methods to
enable the quantification of monetary values. Two essential methods are
important here:

Using \textbf{stated preference methods}, actors can be asked about
various future scenarios. Previously unknown circumstances can thus be
weighed against each other.

In \textbf{revealed preference methods}, a utility assessment for the
future is derived from decisions that have already been made. This
prevents actors from giving strategic answers when assessing expected
future scenarios. However, conditions that have not previously occurred
cannot be assessed.

Although there are established methods for assessing preferences, these
are limited in their ability to weigh up costs and benefits. As the term
``preference'' already suggests that these are always subjective views.
The monetary assessment derived from them will therefore always be
subjective, i.e., not an objective figure.

\textbf{Example:} At the beginning of this millennium, it emerged that
the tobacco company Philip Morris had commissioned a cost-benefit
analysis of the effects of smoking on the Czech Republic. The authors
concluded that smoking generated an annual profit of around 5.8 billion
koruna (approx. USD 150 million) for the Czech state budget. Due to the
lower life expectancy of smokers, the state had to provide for them for
a shorter period of time and therefore saved money.

This report sparked a public outcry, prompting the tobacco company to
apologize. But this example also highlights the subjective nature of
cost-benefit analysis. By reducing the value of human life to a number,
the company had not only crossed a moral line. As a scientist later
showed, the company had also made false assumptions and omitted
far-reaching costs in the report. Her cost-benefit analysis of the same
scenario concluded that smoking in the Czech Republic contributed to an
annual deficit of around 14 billion koruna.

Another problem is that a cost-benefit analysis implies that different
types of capital are substitutable, i.e., that the loss of biodiversity
can be replaced by technology. This assumption is controversial and is
one of the central points of contention between different understandings
of sustainability. We will return to this in section 6 in the discussion
of strong and weak sustainability.

\subsection{Discounting -- the factor of
time}\label{discounting-the-factor-of-time}

The factor of time is included in the cost-benefit analysis by means of
discounting. Costs incurred in the future are valued at a lower level.
The discount rate influences how much perceived costs decrease over
time. Various decision-making criteria can be used to determine the
discount rate. Intergenerational justice, the choice between strong and
weak sustainability (introduced in section 6), uncertainty about future
events, and the time preference of decision-makers influence how much
future costs are discounted.

\begin{figure}[H]

\centering{

\pandocbounded{\includegraphics[keepaspectratio]{images/Fig2_8_discountrate.png}}

}

\caption{\label{fig-discountrate}Illustration of different discount
rates over time}

\end{figure}%

The figure above shows different discount rates over the same period.
For example, the cost of climate damage of \$1,000 occurring in 10 years
is valued at approximately \$860 at a discount rate of 1.5\%. At a
discount rate of 5\%, only around \$600 is included in the cost-benefit
analysis as perceived costs. 10\% discount rate value the \$1,000 in 10
years at around \$350 in the present.

\textbf{Example:} In 2006, British economist Nicholas Stern published
the
\href{https://webarchive.nationalarchives.gov.uk/ukgwa/20100407172811/https:/www.hm-treasury.gov.uk/stern_review_report.htm}{Stern
Report} for the British government. In this comprehensive document, he
describes climate change as the greatest market failure in history and
uses cost-benefit analysis to argue that one percent (he later argued
for two percent) of global GDP must be invested in reducing emissions.

Economists had mixed reactions to Stern's report. While some supported
him, others argued that he had miscalculated the costs of global
warming. The main argument was his low discount rate of 0.1 percent,
which he used to assess future costs. Economist William Nordhaus took
Stern's calculations and changed only the discount rate from 0.1\% to
3\%. According to this, investments to combat global warming are still
necessary, but much lower than Stern demanded. In the logic of the
cost/benefit analysis, Nordhaus concludes that, in the best-case
scenario, global warming will rise to over 3 degrees by 2100 (Nordhaus
(2019)).

The question of the discount rate represents a clear ethical position in
this example. Stern chose a discount rate of 0.1\%, assuming that the
annual risk of human extinction is 0.1\%. Otherwise, the welfare of
future generations should be weighted exactly the same as today's.
Nordhaus' discount rate, on the other hand, weights the welfare of
future generations by 3\% less each year. This is based on the
assumption that the economy will continue to grow at this rate and that
the costs will therefore be perceived as lower.

The calculation of future costs and benefits therefore depends on many
assumptions and ethical questions. These are supposed to be reflected
numerically in the discount rate.

\begin{tcolorbox}[enhanced jigsaw, breakable, colbacktitle=quarto-callout-tip-color!10!white, left=2mm, bottomtitle=1mm, colback=white, rightrule=.15mm, colframe=quarto-callout-tip-color-frame, bottomrule=.15mm, opacityback=0, titlerule=0mm, opacitybacktitle=0.6, toptitle=1mm, toprule=.15mm, title={Questions for reflection (.unnumbered)}, arc=.35mm, leftrule=.75mm, coltitle=black]

\begin{itemize}
\tightlist
\item
  Consider how the following decision-making principles affect the
  discount rate:

  \begin{itemize}
  \tightlist
  \item
    Intergenerational justice
  \item
    High time preference (preference to consume goods in the present)
  \end{itemize}
\end{itemize}

\end{tcolorbox}

\section{Prosperity in a Changing
World}\label{prosperity-in-a-changing-world}

In our modern world, characterized by rapid change and a constantly
growing population, the term ``prosperity'' is often associated with
economic and monetary wealth. However, in times when environmental
change, social inequality, and resource scarcity are becoming pressing
challenges, this conventional view of prosperity is no longer
sufficient. The complexity and multifaceted nature of the concept make
it controversial. In addition, measuring prosperity is extremely
difficult and therefore leads to controversial discussions. There are
many different indices that attempt to measure a country's prosperity.
The most widely used is gross domestic product (GDP). It is omnipresent
in everyday life, politics, and the media and appears to be the most
important economic indicator. GDP is criticized primarily for its narrow
focus on wealth in terms of goods and services sold on the market, and
also because the indicator does not provide a comprehensive picture.

\subsection{Gross domestic product (GDP) as an indicator of
prosperity}\label{gross-domestic-product-gdp-as-an-indicator-of-prosperity}

GDP is considered the main indicator of a nation's prosperity. It
measures the total value of all goods and services produced in a country
in a year after deducting all intermediate inputs. The concept of GDP in
its current form was first developed in the 1930s -- the first attempts
to measure a country's prosperity date back to the work of William Petty
in the 17th century -- to combat the Great Depression and plan the war
economy during World War II in the US and England (Schmelzer and Vetter
(2023), pp.~57-59). A prerequisite for this was the invention of the
``economy'' as an independent sphere of social life that can be
statistically recorded and measured (Schmelzer and Vetter (2023),
p.~57). Despite its widespread use today, GDP is still subject to
criticism. Even economists who promoted the development of GDP warned
against using it as a general measure of prosperity. Simon Kuznets, one
of the most prominent economists in this field, said\emph{:}

\begin{quote}
``Whatever one may do with them, such numbers {[}\ldots{]} appear to be
highly useful; they appear to meas-ure something clearly defined and
significant, making it comparable. But upon closer inspection, the
impression that such estimates are clear and unambiguous proves
misleading.''
\end{quote}

\begin{quote}
(nach Lepenies 2013: S. 90), own translation)
\end{quote}

The main criticism is that GDP only measures the monetary value of goods
and services that have been purchased with money. The entire unpaid
sector of the economy, such as unpaid care work or undeclared work, is
not taken into account. GDP also says little about the actual well-being
of a population, as it is reduced to purely material possessions.
Furthermore, GDP neglects the social and environmental impacts of
economic activities, such as social inequality and environmental
pollution. Damage to the environment can even have a positive effect on
GDP if, for example, after a natural disaster reconstruction has to be
finance, leading to an increase in GDP. Personal accidents also increase
GDP due to medical treatment, even though they represent a significant
deterioration in the well-being of the person concerned. Furthermore,
income distribution is not taken into account, which means that a high
GDP does not necessarily reflect a fair distribution of income. In
addition to these major well-known criticisms, there are several others
that we will not go into further here. Finally, GDP is based on annual
growth rates, which can lead to short-term thinking, with policymakers
and companies pursuing short-term goals to increase GDP rather than
striving for long-term sustainability goals.

The following chart shows the development of Switzerland's GDP compared
with selected other countries over the last 70 years.

\begin{figure}[H]

\centering{

\pandocbounded{\includegraphics[keepaspectratio]{images/Fig2_9_GDP.png}}

}

\caption{\label{fig-GDP}Source: Our World in Data}

\end{figure}%

\subsection{Alternative indicators of
prosperity}\label{alternative-indicators-of-prosperity}

In view of these limitations of GDP, several alternative indicators have
been developed to provide a more comprehensive picture of prosperity.
One of these indicators is the Human Development Index (HDI), which
takes into account not only GDP but also life expectancy and the level
of education of the population. Another indicator is the Genuine
Progress Indicator (GPI). The GPI goes one step further by attempting to
include not only positive economic activities but also negative factors
such as environmental pollution and social inequality in its
calculations. The GPI usually consists of 26 indicators that attempt to
cover the economic, social, and ecological dimensions. For example, the
value of unpaid work is also included. Since it is fundamentally
difficult or even impossible to measure prosperity, these indicators are
also criticized for their limitations, and there are many other
indicators for measuring a country's prosperity. However, none of them
have been able to establish themselves alongside GDP.

\begin{figure}[H]

\centering{

\pandocbounded{\includegraphics[keepaspectratio]{images/Fig2_10_GPI.png}}

}

\caption{\label{fig-GPI}The diagram shows the various aspects that the
GPI attempts to integrate. Source: GNHUSA.}

\end{figure}%

To illustrate how widely different indicators can diverge, the
calculations by Kubiszewski et al.~(2013) for the development of GDP and
GPI per capita since the 1950s for a group of 17 countries are shown
here.

\begin{figure}[H]

\centering{

\pandocbounded{\includegraphics[keepaspectratio]{images/Fig2_11__GDPxGPI.jpg}}

}

\caption{\label{fig-GDPGPI}Own illustration, based on Kubiszweski
(2019)}

\end{figure}%

\subsection{Economic growth}\label{economic-growth}

Economic growth basically refers to the increase in a country's economic
performance or prosperity over time. As a rule, GDP growth is used as a
measure of this. The central importance of economic growth for our
society and economy is omnipresent in everyday life, politics, and the
media. It is the central target for politics, society, and the economy,
which is why Schmelzer and Vetter also refer to it as the growth
paradigm (Schmelzer and Vetter (2023), pp.~62--68). Given the ubiquity
of economic growth, it is surprising that it is a very recent
phenomenon. It was not until the Industrial Revolution in England in the
19th century that sustained economic growth (measured in terms of GDP)
was observed for the first time in the world. For most of human history,
people lived without sustained economic growth. Today, economic growth
is not only a central pillar of our current economic and social system,
but has also become a compulsion (Binswanger (2019)). We will discuss
this systemic compulsion to grow and the growth paradigm in more detail
later in the course. The significance of economic growth is the subject
of intense debate within the sustainable economy, with widely divergent
positions. While some want to hold on to economic growth (in an adapted
form), others advocate overcoming economic growth as a compulsive
behavior. We will examine these discussions in greater depth at a later
stage.

The following chart shows the development of GDP per capita for various
countries and regions over the last 1,000 years. The industrial
revolution in the 19th century is clearly visible.

\begin{figure}[H]

\centering{

\pandocbounded{\includegraphics[keepaspectratio]{images/Fig2_12__GDP1000_2000.png}}

}

\caption{\label{fig-GDPhist}Own illustration, based on Kubiszweski
(2019)}

\end{figure}%

\begin{tcolorbox}[enhanced jigsaw, breakable, colbacktitle=quarto-callout-tip-color!10!white, left=2mm, bottomtitle=1mm, colback=white, rightrule=.15mm, colframe=quarto-callout-tip-color-frame, bottomrule=.15mm, opacityback=0, titlerule=0mm, opacitybacktitle=0.6, toptitle=1mm, toprule=.15mm, title={Questions for reflection (.unnumbered)}, arc=.35mm, leftrule=.75mm, coltitle=black]

\begin{itemize}
\tightlist
\item
  Do you think GDP is a useful measure of a nation's prosperity? What
  are the arguments for and against it?
\item
  Think of examples from everyday life in which the growth paradigm can
  be seen.
\item
  What could be the reasons for the large gap between GDP and GPI? And
  what could be the reason for the trends in the figure diverging so
  sharply only from the 1970s onwards?
\end{itemize}

\end{tcolorbox}

\begin{tcolorbox}[enhanced jigsaw, breakable, colbacktitle=quarto-callout-note-color!10!white, left=2mm, bottomtitle=1mm, colback=white, rightrule=.15mm, colframe=quarto-callout-note-color-frame, bottomrule=.15mm, opacityback=0, titlerule=0mm, opacitybacktitle=0.6, toptitle=1mm, toprule=.15mm, title={Further readings}, arc=.35mm, leftrule=.75mm, coltitle=black]

Schmelzer, Matthias. 2016. The hegemony of growth: the OECD and the
making of the economic growth paradigm. Cambridge: Cambridge University
Press.

Sen, Amartya. 2011. Development As Freedom. New York: Knopf Doubleday
Publishing Group.

\end{tcolorbox}

\bookmarksetup{startatroot}

\chapter{Pluralist economics}\label{pluralist-economics}

As the first sections have shown, economics is currently dominated by a
single school of thought, neoclassical economics. However, economics
consists of many \textbf{different schools of} thought based on
different assumptions, with different perspectives and focuses, some of
which use different methods and arrive at different conclusions. In
order to meet the complexity of current challenges, we believe that it
is essential to be able to draw on this diversity of theories and tools.
This short module therefore aims to provide an introduction to the
diversity of economics, known as pluralist economics.

In the course, there will be some opportunities to explore individual
schools of thought in greater depth. However, it will not be possible to
cover all schools of thought in depth. We are also happy to provide
further material if you would like to explore pluralist economics or
individual schools of thought in greater depth.

This section is not about knowing the details of the different schools
of thought, but rather about becoming familiar with the diversity of
economics. Accordingly, the section mainly provides material to explore
the various schools of thought freely and even playfully. For this
purpose, we created a crossword puzzle to explore the diversity of
economics. You will find it later in the section.

\textbf{Learning Outcomes}

\begin{itemize}
\tightlist
\item
  Students can explain the fundamentals of pluralist economics and
  identify different schools of thought.
\end{itemize}

\section{Neoclassical economics}\label{neoclassical-economics}

As already indicated several times, current economic theory is strongly
dominated by one school of thought, neoclassical economics. In the
following, we will take a brief look at the basic principles of
neoclassical economics and what is problematic about it from the
perspective of a sustainable economics. Criticism of neoclassical
economics should not be equated with the idea that neoclassical
economics as a school of thought is obsolete. Rather, it is important to
recognize the questions for which the neoclassical lens is useful and
the questions for which other schools of thought offer more helpful
perspectives. Furthermore, \textbf{neoclassical economics} is
\textbf{not a homogeneous field,} remains a vague term, and is therefore
sometimes used in different ways. The term was coined by Thorstein
Veblen in the early 20th century to describe Alfred Marshall's synthesis
of subjective and objective value theory as a supply and demand diagram.
Marshall combined the classical understanding that the value of a good
is determined by its production costs with the new insights of the
marginal utility school, which determined value by individual utility.
The market diagram, which represents supply and demand, is at the heart
of neoclassical economics.

According to neoclassical understanding, the central problem of
economics lies in the \textbf{scarcity of societal resources}, because
it is assumed that human needs are fundamentally insatiable. Economics
as a science therefore aims to study the functioning of a national
economy in order to enable the highest possible level of prosperity
through optimal resource allocation. Neoclassical theory bases its
conception of the economy on rational individuals who are faced with
alternatives that need to be weighed up and who want to maximize their
own utility with their decisions. This weighing up of alternatives is
based on a comparison of marginal utility and marginal costs. This
involves the abstraction of \textbf{homo economicus}, who represents an
ideal type of individual who acts rationally in the sense of maximizing
utility and primarily with a view for his own self-interest. Based on
this starting point of weighing up different alternatives, neoclassical
economics is often understood as \textbf{the science of
decision-making}, which makes it applicable to all areas of life. As we
have already seen in the module on the basics of economics, the price
mechanism in a market organizes the allocation of scarce resources and
ensures that supply and demand, assuming perfect competition (see how
the market works), settle at an equilibrium, which also corresponds to
the economic optimum. From this perspective, the \textbf{market} emerges
as the ideal form of allocation and state intervention as disruptive,
which is expressed politically in the liberalization of markets and the
reduction of the state to a minimum.

As already indicated several times, these assumptions underlying
neoclassical economics, such as rational and utility-maximizing
individuals and perfect competition, have been criticized from various
voices. However\textbf{,} much more serious than the explicit
assumptions are the \textbf{implicit value judgments and predetermined
perspective} that are adopted when looking through neoclassical glasses.
For example, entire sectors of the economy, such as the unpaid sector,
are ignored. The idea of what economics is and what problems it
addresses (allocation of scarce resources, focus on decision-making,
etc.) is based on a normative foundation and defines the framework of
possibilities. For example, markets are assumed to be natural, and the
structural inequalities associated with them are accepted as given
conditions. Improvements can only be achieved through marginal changes
to certain target indicators within this framework.

These brief paragraphs on neoclassical economics and some of the
criticisms raised are not intended to provide a comprehensive
description, but merely to show that neoclassical economics takes a
certain perspective (a more detailed description of neoclassical
economics can be found here). We will examine specific aspects in more
detail later on. Given its strengths and limitations, neoclassical
economics is well suited to analyzing some issues, but less so for
others. Economics offers a wide range of thought styles, which are
suited to analyzing different problems to varying degrees.
\textbf{However,} as will be explained below, \textbf{economics} is
\textbf{dominated by neoclassical economics,} which prevents the
economic toolkit from being used to its full potential. In this course,
we also want to open \textbf{your perspective to this diversity}.

\section{Why do we need pluralist
economics?}\label{why-do-we-need-pluralist-economics}

In November 2011, dozens of students at Harvard University walked out of
the introductory lecture given by Professor N. Gregory Mankiw in protest
against what they saw as the politically biased presentation of economic
relationships by lecturers. The students criticized Mankiw's teaching at
Harvard as being characterized by an unacceptable ideological bias that
contributes to the great economic inequality in today's society. This
student protest made headlines not only because of Harvard University's
prominent position in the international academic world, but also because
Gregory Mankiw is the author of an introductory textbook whose enormous
influence on university education in economics has long since reached
global proportions. The students also considered it particularly worthy
of criticism that alternative economic approaches are practically
non-existent in teaching (Harvard Political Review 2011).

In stark contrast to other social sciences such as sociology or
political science, university education in economics today is
characterized by a \textbf{strong paradigmatic uniformity of teaching
content}. Many textbooks do not seek to acknowledge competing paradigms
with different perspectives on economic relationships, but instead too
often suggest to readers that there is a \textbf{single economic way of
thinking,} neoclassical economics, which can and should be applied
equally to all areas of economics---and, in principle, to all social
sciences.

\begin{quote}
``Indeed, I have come to the position that the economic approach is a
comprehensive one that is appli-cable to all human behavior, be it
behavior involving money prices or imputed shadow prices, repeated or
infrequent decisions, large or minor decisions, emotional or mechanical
ends, rich or poor persons, men or women, adults or children, brilliant
or stupid persons, patients or therapists, businessmen or politicians,
teachers or students.'' - Becker, S. Gary in The Economic Approach to
Human Behavior (1976), p.~8
\end{quote}

\href{https://www.nobelprize.org/prizes/economic-sciences/1992/press-release/}{Gary
S. Becker received the Alfred Nobel Memorial Prize in Economic Sciences
in 1992} for his extension of microeconomic theory to a wide range of
human behavior and human cooperation. Critics describe this extension of
microeconomic theory to almost all areas of human behavior and
cooperation as economic imperialism. The sometimes very one-sided
textbooks in economics contribute to economic imperialism. Graupe (2015)
describes economic education as an intellectual monoculture.

Two works are particularly significant in this context: ``Economics'' by
Paul A. Samuelson (co-edited with William D. Nordhaus since 1985) and
``Principles of Economics'' by N. Gregory Mankiw. Samuelson's textbook,
first published in 1948, is considered the standard work on which all
other textbooks today are based. Microeconomics in particular can be
seen as an area ``where the victory of Samuelson's early pedagogy has
been most complete and where the beliefs of economists have changed
least'' (Skousen 1997, p.~138). Mankiw's ``Principles of Economics'' has
become an international bestseller in recent years. This textbook by
George W. Bush's former economic advisor also occupies a central
position at Swiss universities. The main criticism of standard works on
economics is that concepts and the thinking underlying the models and
assumptions are hardly described in concrete terms, threatening to
freeze students' perspectives into a single point of view. Students are
not encouraged to reflect on the conclusions drawn in the textbooks.

\subsection{Blind spots in economics
textbooks}\label{blind-spots-in-economics-textbooks}

After reading Samuelson/Nordhaus or Mankiw, hardly any student will be
able to say exactly how neoclassical economics thinks about ``the
market,'' what ``thinking tools'' it uses to do so, and why it does so.
Nevertheless, through constant repetition requiring the same type of
problem-solving in varying degrees of complexity, students are led to
always think in terms of price-quantity diagrams and to determine
equilibrium quantities and prices, regardless of the specific task at
hand. Whether the questions concern ice cream, wheat, oil, gasoline,
rent, wages, steel, aluminum, education, land, or capital, the textbooks
always claim that they can be solved with the help of this diagram. What
is rarely questioned, however, is the existence of the underlying
functional relationships themselves. Whether ``supply curve,'' ``demand
curve,'' or ``equilibrium price,'' students learn to use these
``thinking tools'' to ponder everything in the world without reflecting
on them themselves and without being able to question the underlying
system of rules and assumptions. As explained in the section on
neoclassical economics, this perspective is shaped by value judgments
and worldviews. Our aim for this course on sustainable economics is to
give students an insight into different thought styles in economics and
enable them to identify the most common arguments in the current
economic and social debate on economic growth, welfare, environmental
protection, and employment. \textbf{This enables students to perceive
the (economic) world in a more differentiated way.} After all, how we
learn to perceive the world also determines what options for action we
recognize and exercise in our everyday lives and as citizens.

\begin{quote}
``I suppose it is tempting, if the only tool you have is a hammer, to
treat everything as if it were a nail.'' - Abraham Maslow in Psychology
of Science 1966, p.~15
\end{quote}

In the video below, Professor Ha-Joon Chang from the University of
Cambridge shows the different thought styles within economics that exist
and how these can lead to different economic policy programs:

\url{https://www.youtube.com/watch?v=NdbbcO35arw}

\section{Scavenger hunt through pluralist
economics}\label{scavenger-hunt-through-pluralist-economics}

Exploring Economics is an open project of the pluralist economics
network, which cooperates with various international actors in the
design of the e-learning platform. The network consists of many local
groups and individuals. There are numerous opportunities to get involved
and contribute.

We believe that this open platform offers many valuable contributions.
That is why we have designed a scavenger hunt through the platform so
that you can get to know it in a playful way.

\textbf{Exercise - Pluralist Economics:} In this section, you will solve
the crossword puzzle
\href{https://ilias.unibe.ch/goto_ilias3_unibe_file_3346159.html}{``A
journey through the Economic Theories''} using the
\href{https://www.exploring-economics.org/en/}{Exploring Economics
platform}. Under the heading ``Orientation,'' you will find solutions to
the terms sought in the puzzle. This will give you an overview of the
different schools of thought in economics and how these perspectives
influence research and its results. This exercise is voluntary and is
intended to give you a playful introduction to pluralist economics. You
are welcome to contact us to discuss any questions or clarify any
uncertainties. On demand we provide you with the solutions of the
crossword puzzle.

\begin{figure}[H]

\centering{

\pandocbounded{\includegraphics[keepaspectratio]{images/Fig3_1_PluralEcon.png}}

}

\caption{\label{fig-pluralecon}Excerpt from the ``Exploring Economics''
website, which categorizes the different schools of thought according to
the question ``Which problem is central to the economy?''.}

\end{figure}%

\begin{tcolorbox}[enhanced jigsaw, breakable, colbacktitle=quarto-callout-note-color!10!white, left=2mm, bottomtitle=1mm, colback=white, rightrule=.15mm, colframe=quarto-callout-note-color-frame, bottomrule=.15mm, opacityback=0, titlerule=0mm, opacitybacktitle=0.6, toptitle=1mm, toprule=.15mm, title={Further reading}, arc=.35mm, leftrule=.75mm, coltitle=black]

If you are already well acquainted with the basics, or if you wish to
engage more deeply with neoclassical economics, the following literature
is recommended for further study:

\begin{itemize}
\tightlist
\item
  The first chapter of the book ``The Microeconomics Anti-Textbook'',
  which examines how economics is presented in introductory
  microeconomics textbooks. This chapter is particularly valuable for
  economics students already familiar with the basics, as it provides a
  contrast to the standard textbook.
\item
  For interested students, the critical discussion of neoclassical
  consumption theory by Ben Fine (2016) is recommended
\end{itemize}

\end{tcolorbox}

\bookmarksetup{startatroot}

\chapter{Problem analysis}\label{problem-analysis}

\textbf{Learning Outcome}

Students\ldots{}

\begin{itemize}
\item
  be able to identify key ecological, social, and economic problems of
  the current economic and social system
\item
  be able to explain at least one key problem (ecological, economic, or
  social) using an example
\end{itemize}

\section{The world in transition -- a
polycrisis}\label{the-world-in-transition-a-polycrisis}

The current economic system is leading to major challenges that are
manifesting themselves in various crises. This current multitude of
crises is recently referred to as a \textbf{polycrisis} (see, for
example, Lawrence et al.~2024 or this
\href{https://www.weforum.org/videos/experts-explain-adam-tooze-what-is-the-polycrisis/}{explanatory
video} by Adam Tooze) because they affect people's working and living
environments in many different ways and are intertwined. Some of the
individual crises are closely linked and have common causes. Others
appear to be more independent of each other and do not necessarily have
the same causes. Taken together, however, all these crises culminate in
this often overwhelming polycrisis. In this section, we will
\textbf{analyze these crises} before looking at solutions for a
sustainable economy. In the following, we will address the ecological,
social, and economic challenges separately. However, it is important to
emphasize that these dimensions are strongly interlinked and can
reinforce each other. Separating these areas is intended to facilitate
the introduction.

As we have seen in the previous sections, the analysis itself is shaped
by the perspective of the person conducting the analysis. The problem
analysis can be different depending on the school of thought. For
example, neoclassical economics doesn't necessarily see rising income
and wealth inequality as a problem as long as it's caused by market
mechanisms. A school of thought that includes power aspects in its
analysis, such as feminist or Marxist economics, sees this inequality as
a central problem for the economy and society. In line with the focus of
this course, we will keep the problem analysis as broad as possible,
covering all schools of thought.

\section{Ecological challenges}\label{ecological-challenges}

\pandocbounded{\includegraphics[keepaspectratio]{images/Fig4_1__greatacceleration.png}}In
the wake of the Industrial Revolution, a uniquely productive mode of
production emerged, enabling a massive increase in material prosperity.
At the same time, but much less noticed, there was also a massive
increase in the consumption of natural resources, environmental
pollution, and emissions (Jarrige and Le Roux (2020)). Socio-economic
growth went hand in hand with the acceleration of biophysical trends.
The description of exponential growth dynamics is called ``the great
acceleration'' (Steffen et al. (2015)). The graph above shows some
important biophysical and socioeconomic indicators that all began to
rise with the Industrial Revolution. From the middle of the 20th
century, the trend toward exponential growth becomes apparent.

\begin{tcolorbox}[enhanced jigsaw, breakable, colbacktitle=quarto-callout-note-color!10!white, left=2mm, bottomtitle=1mm, colback=white, rightrule=.15mm, colframe=quarto-callout-note-color-frame, bottomrule=.15mm, opacityback=0, titlerule=0mm, opacitybacktitle=0.6, toptitle=1mm, toprule=.15mm, title={Brief digression: exponential growth (optional reading for interested
readers)}, arc=.35mm, leftrule=.75mm, coltitle=black]

The exponential function is used to describe the size of things that are
subject to accelerated growth. A crucial step is the realization that
even moderate growth of, say, five percent means exponential growth,
i.e., the increase becomes constantly larger (this applies, for example,
to economic growth or to the number of cases during the COVID-19
pandemic). Exponential growth is difficult for us humans to understand.
At the beginning, it is quite slow. This is precisely why it is often
greatly underestimated. However, the growth rate then increases
continuously and eventually becomes enormous. An insightful value that
quickly reveals growth processes is the doubling time, which can be
easily calculated in your head: 70 divided by the percentage growth per
unit of time. If something grows by seven percent per year, the doubling
time is ten years; at five percent, it is fourteen years; and at one
percent growth, it is seventy years. The number 70 is the result of
multiplying 100 (the percentage value of the doubling) by the natural
logarithm of 2. It is not necessary to understand how this number is
derived. The only important thing is: 70 divided by the percentage
growth.

The following thought experiment also helps to understand this better:\\
A client offers you two different fees for a 64-day assignment. Which
fee do you choose?

A: You receive 10,000 CHF every day.\\
B: You receive 1 cent on the first day and the amount is doubled on each
subsequent day.

Thought experiment: the Payout matrix

\end{tcolorbox}

\subsection{Planetary boundaries and the climate
crisis}\label{planetary-boundaries-and-the-climate-crisis}

Since the late 18th century, when British economist Thomas Robert
Malthus first proposed a theory of overpopulation in his Essay on the
Principle of Population, the discourse on planetary carrying capacity
has persisted in certain scientific theories. Even though Malthus' ideas
have since been proven obsolete -- he calculated the maximum population
based on a fixed amount of food that could be produced -- the basic idea
has returned in the form of neo-Malthusianism. In the study
``\textbf{The Limits to Growth}'' published by the Club of Rome in 1972,
a population limit was calculated using a neo-Malthusian approach based
on the amount of food available, but also taking into account available
resources, environmental pollution, and industrial output. Although the
predicted developments have not materialized, the study still carries
weight today from an ecological perspective that questions unlimited
growth.

The latest theoretical development in this regard is the scientific
concept of \textbf{planetary boundaries}. Unlike previous concepts,
these do not refer to a maximum population size, but to parameters
derived from Earth system science. Researchers have chosen the state of
the Earth system in the Holocene as their starting point. In this era,
the planet had the ideal conditions for human civilizations to emerge.
Deviations from this state put humanity in an uncertain territory where
tipping points can be reached. Exceeding these tipping points could
either halt current developments, change their direction, or accelerate
them significantly. One example of this is the extinction of many large
mammals at the end of the last ice age as a result of human immigration
to the American continent. The concept of planetary boundaries
recommends applying the precautionary principle in order to reduce
potential damage to humans and the environment. The current planetary
health report uses planetary boundaries as a basic framework for
evaluating the health of our planet (Ceasar et al. (2024)). In addition
to the report, the recently launched
\href{https://www.planetaryhealthcheck.org/}{Planetary Health Check
initiative} offers exciting opportunities on its website to explore
planetary boundaries and their development over the past 70 years.

\begin{figure}[H]

\centering{

\pandocbounded{\includegraphics[keepaspectratio]{images/Fig4_2_PlanetaryBoundaries.png}}

}

\caption{\label{fig-planbound}Planetary boundaries of the various
subsystems and the respective risk assessment of the current status.
Green indicates the safe operating space, yellow indicates increased
uncertainty. Beyond yellow indicates that the risk of fatal consequences
is considered high. Source: Planetary Boundaries Science (PBScience)
2025 licensed under under CC BY 4.0}

\end{figure}%

The planetary boundaries refer to nine different subsystems, such as
land use change and ocean acidification. An inner circle (safe operating
space) and an outer circle (increased uncertainty) have been defined for
each of these. Six of the nine planetary boundaries have already been
exceeded. Ocean acidification is close to being exceeded, while aerosol
pollution is falling slightly again. Trends in stratospheric ozone
levels are pointing in slightly different directions. The degree of
exceedance has in-creased for all previously identified limits (Ceasar
et al. (2024)).

Within these nine subsystems, there are two so-called core boundaries:
the integrity of the biosphere and climate change. These two systems
combine the processes of many other subsystems and have an impact at the
supraregional level. Reaching tipping points in these two systems can
therefore cause the entire Earth system to enter a new state.

Rapid climate change in particular poses a major challenge for humanity
in the 21st century. Because the energy system, transport
infrastructure, and industrial agriculture are based on fossil fuels
such as oil and gas, excessive greenhouse gases such as carbon dioxide
and nitrogen oxides are continuously emitted. These accumulate in the
atmosphere and prevent the sun's heat, which enters the atmosphere
through solar radiation, from escaping again (greenhouse effect). As a
result, glaciers and ice sheets have shrunk, oceans have warmed, and sea
levels have risen over the past 30 years. Extreme temperature anomalies
and precipitation events are also increasing continuously. Today, the
concentration of greenhouse gases in the Earth's atmosphere is the
highest it has been in the last 800,000 years, and the global average
temperature has risen by more than one degree Celsius since the
Indus-trial Revolution.

In order to avoid reaching climate tipping points, the Paris Agreement
of 2015 set a target of limiting global warming to two degrees Celsius.
The aim is to keep the average temperature rise since the Industrial
Revolution ``well below 2 degrees Celsius.'' Even with a two-degree rise
in temperature, massive biophysical changes are likely. And to achieve
this goal, the energy supply must completely phase out fossil fuels by
2050. Such a transition poses major challenges for society, particularly
because fossil fuels have been a key factor in the development of the
current economic system and have a strong influence on its structure
(see, for example, Malm (2016), Huber (2013), Kallis and Sager (2017)).

\section{Social challenges}\label{social-challenges}

Social boundaries and challenges are often less clear to define. They
also depend heavily on the perspective we take. Accordingly, these
aspects are often discussed more controversially than, for example,
planetary boundaries. The following section highlights some aspects that
are central to the social dimension of sustainable development. The
explanations are not exhaustive, but are intended to highlight key
challenges in this dimension.

\subsection{Inequality in income and
wealth}\label{inequality-in-income-and-wealth}

At the beginning of the 21st century, global poverty dominated the
discussion on global social welfare, in line with the corresponding
Millennium Development Goal. Although poverty still ranks first among
the UN's 17 Sustainable Development Goals (SDGs), inequality has also
made it onto the global agenda with SDG 10, ``Reduce inequality.'' The
targets for this goal cover many forms of inequality, but here we focus
on income and wealth inequality.

In the middle of the last century, US economist Simon Kuznets
hypothesized that in countries with increasing economic growth, income
inequality would initially rise but then decline again once a certain
level was reached. The so-called ``Kuznets curve'' emerged during the
flourishing social market economies of the 1950s and 1960s, when profits
were widely distributed among workers in the growing economies of the
Global North and the highest incomes were taxed at highly progressive
rates. With the advent of neoliberal reforms in the 1980s, income
polarization became increasingly apparent in the US and, to a lesser
extent, in Europe. Since then, the share of total income earned by the
richest 10\% has risen, while that of the bottom 50\% has declined (see,
for example, \href{https://wid.world/}{the World Inequality Database)}.

An important cause of rising income inequality is \textbf{the changing
balance of power between labor and capital in an increasingly open
global economy}. While a large proportion of financial capital can now
be moved around the world in a fraction of a second at one click, the
mobility of workers has increased relatively little. Institutional
factors such as borders, but also social motivations such as family or
friendships, slow down the production factor of labor. The previously
established compromise between companies and employees organized in
trade unions has come under pressure in many places. As a result, the
wage share (proportion of earned income) is falling in many countries
today, while the share of capital income is rising.

\begin{figure}[H]

\centering{

\pandocbounded{\includegraphics[keepaspectratio]{images/Fig4_3_inequality.png}}

}

\caption{\label{fig-inequality}Elephant or hockey stick? Source: Lannen
et al.~2019}

\end{figure}%

At the global level, a similar picture emerges between 1988 and 2008.
The curve showing \textbf{relative income growth (in black)} indicates
that incomes in the middle percentiles of global income have risen
relatively sharply. This reflects the rapid emergence of middle income
groups in countries such as China and Brazil. The lowest quantiles
illustrate how incomes in Africa and South Asia are growing only slowly.
The trough between the richest 80 and 95 percent represents large parts
of North America and Europe, where middle incomes have stagnated since
the neoliberal reforms mentioned above. This representation, known as
the ``Milanovic elephant,'' thus contains information about various
phenomena simultaneously. However, when looking at \textbf{absolute
income growth (in red)} -- also known as the ``hockey stick'' -- it
becomes clear how much the incomes of the richest people have increased
during this period. More than half of all income growth during this
period went to the richest five percent of the world's population. The
poorer half of the world's population received only one-tenth of the
income.

As great as income inequality is, it pales in comparison to wealth
inequality -- i.e., the total value of all assets owned by a person. In
his 2013 book Capital in the Twenty-First Century, French economist
Thomas Piketty showed how wealth inequality has grown again since the
mid-20th century. He sees the main driver behind this as being that
capital income has grown faster than the real economy. As a result, rich
people who can invest in capital goods have access to wealth growth that
is denied to working people. Waves of political liberalization following
the neoliberal shift in the 1980s have also contributed to the increase
in wealth inequality. Low inheritance taxes and the privatization of
state-owned companies, for example, have led to wealth being
concentrated in the hands of a few rich individuals.

\subsection{Unequal globalization}\label{unequal-globalization}

The economic globalization of recent decades (as well as the first wave
of globalization 150 years ago) is often portrayed in a positive light,
highlighting how freer international trade leads to global prosperity.
However, as Milanovic noted more than 20 years ago, this obscures the
\textbf{other side of globalization}, a side marked by exploitation and
destruction (or colonialism) (Milanovic 2003). Globalization is a
phenomenon of enormous magnitude and complexity and can therefore show
both its beautiful and its cruel face (depending on who sees it)
(Milanovic 2003). In addition to the fact that global capitalism has led
to significant growth in material prosperity in many places, an increase
in inequality has also been observed, as explained above. Furthermore,
the economic system in many places was based on colonial and imperial
violence. To this day, these structures lead to exploitation,
dependencies, and unequal exchange in many places (see, for example,
Chang (2002), Anievas and Nişancioğlu (2015), M. Davis (2017), Ghosh
(2021), Hickel (2017)).

Dorninger et al. (2021) shows that high-income countries consume more
resources, land, energy, and labor than they produce themselves. In this
respect, there is an \textbf{unequal exchange} between the different
country groups. In addition, the globalized world economy and the
corresponding institutions are pushing many countries in the Global
South into the production of raw materials (see, for example, Palma
(2003)). At the same time, many countries are active in highly
specialized industries and services, where more value can be extracted.
Global value chains are often controlled by companies in the Global
North, which reinforces this unequal exchange and enables these
companies to skim off the lion's share of the profits and shift the
costs to the Global South (see, for example, Carballa Smichowski,
Durand, and Knauss (2021), Althouse et al. (2023), Durand and Milberg
(2020), Ponte (2022)). Such constellations increase pressure on working
conditions and favor exploitative relationships in the Global South.

This is only a small and simplified excerpt from the complexity of the
globalized economy. Many aspects are still insufficiently researched and
controversially discussed. However, the main point here is to show that,
as Milanovic argues, globalization has at least two faces. Which one it
shows must be examined case by case.

\subsection{Social acceleration}\label{social-acceleration}

As the graphs by Steffen et al. (2015) show under ecological challenges,
not only have earth system trends accelerated rapidly in the 20th
century, but so have socio-economic trends. Hartmut Rosa describes
modernization as a process of ``social acceleration'' (see, for example,
Rosa (2003), Rosa (2016)). According to Rosa, technological progress is
accompanied by accelerating \textbf{social change} (e.g., changes in
social relationship patterns) and an increasing \textbf{pace of life.}
These acceleration processes are driven by economic, cultural, and
structural factors and lead us to feel increasingly pressed for time and
stressed, despite constant technological progress and the associated
efficiency gains. For example, enormous advances in communication
technologies have enabled us to communicate much more efficiently over
long distances (e.g., emails instead of letters). Where we used to write
a few letters at most, the number of emails sent has increased
massively, and we fill the time freed up by sending additional emails,
which in turn brings with it further obligations and changes
relationship patterns. This has increased the possibilities and
requirements for availability, for example. Technological progress is
also accelerating the possibilities for experiencing parts of the world
(e.g., through advances in transportation and communication). The world
we can potentially experience is expanding faster than the proportion we
can actually experience. In relative terms, therefore, the proportion of
the world we can experience is shrinking. Such acceleration dynamics
have an impact on the social well-being of individuals and societies and
cannot therefore be ignored.

\section{Economic challenges}\label{economic-challenges}

The economic interrelationships and the associated problems and
boundaries are also complex and less clearly visible and illustrative
than, for example, planetary boundaries. Nevertheless, there are
economic interrelationships that promote crises and limit the scope for
action. Some of these interrelationships will be briefly outlined below.
However, these explanations are by no means exhaustive.

\subsection{Growth dependency}\label{growth-dependency}

The current capitalist economic system is structurally dependent on
growth. This means that a decline in economic activity in the form of
stagnation, recession, or even depression would lead to an economic
crisis with far-reaching consequences for the population (e.g.,
unemployment, impoverishment, cuts in social services) (Schmelzer and
Vetter (2019), p.~26). \textbf{If growth fails to materialize, the
system enters a crisis.} There is either growth or contraction, but
nothing in between. Matthias Binswanger shows that this compulsion to
grow is rooted in market competition (Binswanger (2019)). Companies that
do not make a profit are forced out of the market. If the average profit
across the entire economy is negative, a corresponding number of
companies are forced out of the market, which can lead to a downward
spiral. Furthermore, the system contains many \textbf{growth drivers}
that structurally increase growth dependency (e.g., technological
progress, interest rate system, corporate forms, etc.). Many political
and social institutions are also heavily dependent on growth (e.g.,
social security systems are largely financed by paid work, making full
employment and growth central aspects of its financing). Hartmut Rosa
shows, for example, that culture in the Global North also functions
strongly as a driver of growth and is based on acceleration (Rosa
(2003)).

The concrete effects of growth, for example on society and the
environment, are also controversial in some respects and are reflected
in the debate between green growth and post-growth (we will examine this
debate in more detail in section 5 and 6). The key point here, however,
is that dependence on growth severely limits the scope for action.
Accordingly, there are many approaches that primarily seek to reduce the
system's dependence on growth.

\subsection{Financialization and financial
crises}\label{financialization-and-financial-crises}

Before 2008, influential representatives of economics were so confident
that they believed they had found instruments for crisis-free management
of market economies. It was believed that recessions and, above all,
depressions could be avoided. The causes of previous crises were said to
be inefficient government regulations and interventions. These had
prevented the rationality of individual market participants from leading
to the market equilibrium. The extensive deregulation of financial
markets was thought to have created self-regulating mechanisms for
economic stabilization. This made the 2008 banking crisis all the more
surprising to most economists. Their belief in the rationality of
markets proved to be a mistake when the bankruptcy of the prestigious
Lehman Brothers bank in September 2008 led to a panic on the stock
markets. The idea that financial markets function efficiently and can be
deregulated accordingly (based on Eugene Fama's market efficiency
hypothesis (1970) and in the tradition of neoclassical economics) was
confronted with reality. It became apparent that post-Keynesian
theories, for example, which point to the possibilities of inherent
instabilities in the system, are carrying important and valid insights.
In particular, the work of Hyman Minsky (1992) received renewed
attention, as he emphasized the inherent instability of the financial
system in his \textbf{financial instability hypothesis}. The worst
effects of the financial crises were prevented because politicians and
experts had learned from the experiences of the Great Depression
following the global crisis in 1929: governments and central banks
immediately took over the function of stabilizing the system (ideas that
go back to the theories of John Maynard Keynes).

The \textbf{basic economic function of the financial sector} is to
provide loans to finance investments (mostly by companies). The
regulation of the financial sector is intended to ensure that the
financial industry primarily fulfills this function in the best possible
way. The strict financial market regulation that followed the stock
market crash of 1929 was an essential foundation of welfare capitalism,
in which activities in the real economy (production in factories,
retail, large infrastructure projects, etc.) were at the center of
economic life. Starting in the 1980s, financial market regulations were
gradually dismantled. This strengthened financial market interests with
their business model of exploiting price changes for internationally
traded assets (stocks, bonds, commodities, derivatives, etc.) to
generate profits. New technological possibilities were exploited, such
as high-frequency trading, in which high-performance computers use
pre-programmed algorithms to react to the smallest price changes within
microseconds and buy or sell assets.

This shift in economic dynamics and power to the financial sector is
called \textbf{financialization}. A power complex emerged in the
financial sector, consisting of central banks, commercial banks, and
other financial institutions, private pension funds, and the owners of
large and small fortunes associated with them. Through financialization,
financial markets grow disproportionately to the real economy, which has
been made possible by national deregulation measures -- exactly in line
with the development observed by Thomas Piketty (see chapter Social
challenges).

Twin Peak Paradox: Debt Crisis and Inequality

In recent years, the link between income and wealth distribution and the
accumulation of debt has come increasingly into focus. Rising income and
wealth inequality can contribute to trade imbalances. This is the case,
for example, when individual countries either stimulate the weakening of
mass purchasing power with credit-financed private demand for
consumption (US and UK) or attempt to compensate for low domestic demand
with rising export surpluses (Germany, China, and Japan). Furthermore,
increasing inequality and growing indebtedness are closely connected.
Economists at the International Monetary Fund (Kumhof and Rancière
(2010)), for example, argue that rising inequality leads to increasing
borrowing among the lower and middle classes to sustain their
consumption levels, while the richest individuals have limited capacity
to spend it on consumption (e.g.~you can have only so many yachts).
Therefore, they invest their money in the financial sector, which gives
out loans to the lower and middle classes. Furthermore, this contributes
to the dynamics of financialization outlined above. In this
\href{https://youtu.be/QHL4GBu-pQQ}{video}, Yanis Varoufakis describes
the current situation of high debt and great inequality as a twin peak
paradox.

\bookmarksetup{startatroot}

\chapter{Strategies and economic policy paradigms for
sustainability}\label{strategies-and-economic-policy-paradigms-for-sustainability}

After having explored ecological, social and economic challenges, this
module turns to strategies and frameworks to address these challenges.
From these more concrete approaches and policies will be derived which
will be explored in section 7.

\textbf{Learning Outcome}

By the end of the module, students\ldots{}

\begin{itemize}
\item
  know the three strategies for sustainability and be able to explain
  them using examples
\item
  know three economic policy frameworks and are able to describe them
\item
  are able to explain their own position on the importance and weighting
  of the different strategies and guiding frameworks for the
  implementation of sustainable development
\end{itemize}

\section{Strategies for
sustainability}\label{strategies-for-sustainability}

In view of the ecological limits of our planet, there is broad consensus
at the scientific and political level in most highly industrialized
countries that planetary boundaries must be respected in the long term.
With the 1.5-degree and 2-degree targets, a central climate policy goal
was also established internationally in the
\href{https://www.fedlex.admin.ch/eli/fga/2017/81/de}{Paris Agreement},
which Switzerland co-signed. With regard to economic and social
challenges, the global community has also committed itself to
the\href{https://sdgs.un.org/2030agenda}{2030 Agenda}, and Switzerland
specifically articulated the goals in
\href{https://www.are.admin.ch/are/de/home/nachhaltige-entwicklung/strategie/sne.html}{the
2030 Sustainable Development Strategy}. These goals require a
fundamental transformation of our economic system within a few decades.
But how exactly should these goals be achieved? There are three basic
strategies for implementing sustainable development that are currently
being discussed: \textbf{efficiency strategy, consistency strategy, and
sufficiency strategy.} All approaches to sustainable development are
based on these three basic strategies, but they differ in how they align
the interaction between the three strategies and in how they prioritize
these strategies.

Various strategies are being discussed for implementing sustainable
development: All three strategies must work together to achieve
sustainability goals.

\begin{figure}[H]

\centering{

\pandocbounded{\includegraphics[keepaspectratio]{images/Fig5_1__NEstrategies.jpg}}

}

\caption{\label{fig-SDstrategies}The three sustainability strategies;
our own representation}

\end{figure}%

\subsection{Efficiency strategy}\label{efficiency-strategy}

The efficiency strategy aims to use fewer resources (raw materials and
energy) than before in the manufacture of products or the provision of
services without reducing the quantity produced. This includes
minimizing material use (material intensity) and energy use (energy
intensity) as well as emissions of harmful substances such as
CO\textsubscript{2}. This concept is often referred to as
\textbf{eco-efficiency} and is considered promising in business and
society because it can reduce costs, resource consumption, and
environmental pollution. The efficiency strategy focuses on the
production side, where change is primarily sought through technological
advances. Proponents of this approach believe that it is possible to
double prosperity while halving the consumption of natural resources
(Weizsäcker, Lovins, and Lovins (1997)). Critics of the efficiency
strategy are less optimistic, however, and warn against overestimating
its impact on sustainable economic activity. They point to so-called
\emph{rebound effects,} which can lead to the gains from efficiency
improvements being reduced or even offset (Paech (2012); Santarius
(2014)).

\begin{tcolorbox}[enhanced jigsaw, breakable, colbacktitle=quarto-callout-tip-color!10!white, left=2mm, bottomtitle=1mm, colback=white, rightrule=.15mm, colframe=quarto-callout-tip-color-frame, bottomrule=.15mm, opacityback=0, titlerule=0mm, opacitybacktitle=0.6, toptitle=1mm, toprule=.15mm, title={Examples of efficiency strategies}, arc=.35mm, leftrule=.75mm, coltitle=black]

Energy-efficient lighting: Replacing conventional light bulbs with
energy-efficient LED lamps re-duces electricity consumption and helps to
lower energy requirements.

Fuel-efficient vehicles: The development of hybrid or electric vehicles
with improved fuel con-sumption reduces fuel demand and emissions from
traffic.

Efficient building technology: Installing smart heating, ventilation,
and air conditioning systems in buildings helps reduce energy
consumption for heating and cooling.

Efficient water use: The use of water-saving systems in households and
industrial companies reduces water consumption and minimizes waste.

\end{tcolorbox}

\subsection{Consistency strategy}\label{consistency-strategy}

While the efficiency strategy is quantity-oriented -- less resource
consumption with more output -- the consistency strategy strives for
compatibility between nature and technology, with the goal of reusing
resources instead of consuming them only once. This implies replacing
materials, products, and technologies that are often based on fossil
resources with those that are compatible with natural material cycles
and run in harmony with natural processes Pufé (2017). This is often
referred to as \textbf{eco-effectiveness} and follows the ``cradle to
cradle'' principle, whereby products go from ``cradle to cradle''
instead of ``cradle to grave.'' The idea behind this is that in
intelligent systems, there is no waste, only products. This can be
achieved in two ways: materials can either be biodegradable, as in
shampoo without synthetic ingredients, or they can be designed as
``technical nutrients'' that remain in the technical cycle. This means
that a product that has reached the end of its useful life does not end
up in the trash, but is transferred to the next cycle of use, for
example by being upcycled. For example, a computer case could be reused
over and over again or converted into a shelving system. The consistency
strategy also focuses on the production side. This strategy is expected
to offer greater problem-solving potential, a wider reach, and more
profound changes than the efficiency strategy.

\pandocbounded{\includegraphics[keepaspectratio]{images/Fig5_2_CircularEconomy.png}}Nevertheless,
material cycles in the economy are not feasible without mass and energy
losses, so absolute consistency remains an unattainable ideal. Even
100\% biodegradable products consume energy in their manufacture.
Nevertheless, consistency is seen as an impetus for industry to strive
for this ideal and to reduce both resource consumption and emissions as
far as possible.

\begin{tcolorbox}[enhanced jigsaw, breakable, colbacktitle=quarto-callout-tip-color!10!white, left=2mm, bottomtitle=1mm, colback=white, rightrule=.15mm, colframe=quarto-callout-tip-color-frame, bottomrule=.15mm, opacityback=0, titlerule=0mm, opacitybacktitle=0.6, toptitle=1mm, toprule=.15mm, title={Examples of consistency strategies}, arc=.35mm, leftrule=.75mm, coltitle=black]

Renewable energy systems: The transition from fossil fuels to renewable
energy sources such as solar, wind, and hydro power ensures that energy
production is in harmony with the natural rhythms and cycles of the
environment.

Recyclable electronics: Electronic devices are designed to be easy to
repair, upgrade, and recycle in order to minimize resource consumption
and environmental impact.

Sustainable construction: Buildings are constructed using sustainable
materials that have a low environmental impact and can be reused or
recycled at the end of their useful life.

\end{tcolorbox}

\subsection{Sufficiency strategy}\label{sufficiency-strategy}

``Sufficiency'' questions the extent of what is needed for a good life
and aims to reduce resource and energy consumption by lowering demand
for resource-intensive goods and services. Sufficiency is often
translated as contentment, frugality or moderation. It takes a critical
look at the new needs created by technology and advertising in relation
to limited natural resources. Sufficiency promotes an understanding of
not chasing every newly created need and to satisfy needs without
commercial consumption. Unlike efficiency and consistency, sufficiency
focuses on consumption, but not exclusively: sufficiency can be
practiced to varying degrees and at different levels, from minor
behavioural changes (sharing instead of buying) to significant lifestyle
changes (giving up air travel). Although it starts at the individual
level, sufficiency can be applied at various levels, including
businesses (sufficiency-oriented product design) and governments
(sufficiency policy). Sufficiency therefore asks about the right amount:
How much do we need for a good life? And what do we not need? The
sufficiency strategy is the subject of passionate and intense debate
(Sedláček (2012)).

Critics consider the sufficiency strategy to be limited in terms of its
savings potential and socio-cultural resonance. They doubt that it can
gain broad acceptance in the population. Proponents, on the other hand,
believe that the sufficiency strategy has an essential place in
sustainability policy, especially where efficiency and consistency
strategies reach their limits.

\begin{tcolorbox}[enhanced jigsaw, breakable, colbacktitle=quarto-callout-tip-color!10!white, left=2mm, bottomtitle=1mm, colback=white, rightrule=.15mm, colframe=quarto-callout-tip-color-frame, bottomrule=.15mm, opacityback=0, titlerule=0mm, opacitybacktitle=0.6, toptitle=1mm, toprule=.15mm, title={Examples of sufficiency strategies}, arc=.35mm, leftrule=.75mm, coltitle=black]

Sharing and communal use: Platforms and initiatives for sharing items,
tools, or vehicles enable resources to be used more efficiently and
reduce the number of products manufactured.

Plant-based diet: Switching to a predominantly plant-based diet reduces
the demand for re-sources such as water and land compared to meat
production.

Shorter working hours: Reducing working hours can lead to lower resource
use, as less energy and materials are needed to produce goods and
services.

Local consumption: Supporting local producers and markets helps reduce
transport distances and emissions.

\end{tcolorbox}

\section{Economic policy paradigms}\label{economic-policy-paradigms}

In order to achieve the sustainability goals, measures based on the
three sustainability strategies must be implemented. How the goals are
achieved, i.e., which combination of strategies is pursued, and which
measures are implemented, is largely determined by the underlying
mindsets and worldviews. However, these mindsets do not exist in
isolation in the economic policy discourse. Rather, many economic policy
approaches are simultaneously shaped by different thought styles.
Following Novy, Bärnthaler, and Prieler (2023), we distinguish between
three basic economic policy paradigms that emerge from the influences of
different thought styles (see Figure~\ref{fig-pluraleconparad}) and
their scientific concepts: the liberal market paradigm, the welfare
capitalist paradigm, and the post-growth paradigm. The economic policy
paradigms and their understanding of sustainability, which underlie the
different approaches, determine to a considerable extent their
orientation and their mix of strategies and policies.

\subsection{Liberal market liberal
paradigm}\label{liberal-market-liberal-paradigm}

The main goal of the liberal market paradigm is to protect individual
freedoms as much as possible and minimize government intervention,
especially in the form of bans and market restrictions. This framework
mainly promotes negative freedoms and relies on a comprehensive market
order that protects and enforces property and contract rights as well as
competition. The idea is that almost all areas of the economy, including
basic services, should be organized according to market principles. This
is intended to enable both the maximum development of individual
freedoms and the efficient use of limited resources.

The liberal market paradigm argues that market regulation is more
efficient than state intervention, even in areas such as environmental
protection or the setting of minimum wages. Although it is recognized
that in a complex, interconnected society, completely free markets,
global trade, companies, and consumers can never be fully realized, the
idea of self-regulating markets serves as a guideline for economic
policy decisions. Such a market order should ensure that companies and
consumers can make individual decisions as freely and independently as
possible. This requires not only state enforcement of a liberal market
order, but also technology neutrality.

The goal of the liberal market paradigm is thus to protect individual
freedoms while ensuring that markets remain as free as possible from
state intervention and neutral with regard to technology.

\subsection{Welfare capitalist
paradigm}\label{welfare-capitalist-paradigm}

The main goal of the welfare capitalist paradigm is to ensure and
protect material prosperity. This approach attempts to combine
competitive economic measures with social justice and, in green
variants, with environmental protection. Based on the successes of the
20th century, it is argued that promoting the capitalist economic system
is compatible with a good life for all.

This has been achieved in part through the establishment of welfare
states, which have largely solved social problems such as poverty and
unemployment by providing basic services---at least in many wealthy
countries. The hope is that a similar approach can also help to overcome
the ecological challenges. The goal is to ensure a secure material
standard of living by increasing production and incomes while reducing
resource consumption and emissions.

Similar to the liberal market paradigm, green growth also plays a
central role in the welfare capitalist approach. This is intended to
enable win-win situations and avoid social tensions.

\subsection{The post-growth paradigm}\label{the-post-growth-paradigm}

The main goal of the post-growth paradigm is a fulfilling life in a
sustainable society that exists in harmony with nature. To achieve this,
it is necessary to overcome the compulsion for constant economic growth
in the capitalist system. The aim is to replace the existing
exploitative relationship between humans and nature with one in which
humans are seen as part of an ecological system.

In order to preserve the foundations of life, the post-growth paradigm
strives for an economy that remains in a stable state (steady-state
economy) or can function independently of economic growth and ensures
the basic provision of human needs. A central principle here is
sufficiency, which means finding the right amount of resource use. This
may mean using fewer resources in some areas to reduce environmental
impacts.

The post-growth paradigm recognizes that not all economic sectors need
to grow and that contraction in some areas does not necessarily
represent a loss. The role of the state in this approach is
controversial, as state actors often maintain unsustainable conditions,
but are at the same time important actors to establish sustainable
frameworks. The emphasis is more on global cooperation and a stronger
role for civil society in reshaping society ``from below.''

These economic policy paradigm -- market liberalism, welfare capitalism,
and post-growth -- represent different views on how society should
respond to pressing issues in the field of environmental protection and
sustainability. They reflect how policymakers and societies at the
global level are trying to stay within planetary boundaries and achieve
climate goals. Since these guiding frameworks are based on scientific
concepts of different schools of thought, they are strongly influenced
by scientific debate. On the other hand, these models also shape
scientific debate, since economics is not free from normative
influences, as we discussed in section 1.

In the next section, we will explore various approaches to sustainable
economics. We will draw on these guiding paradigm, focusing in
particular on the post-growth paradigm and, to a lesser extent, on green
growth approaches within the welfare capitalist paradigm. In line with
the different influences of these schools of thought on the frameworks,
we will also have to consider the various economic theories and
approaches of pluralist economics. These theories offer different
perspectives on the challenges and goals of sustainable economics and
shape the three paradigms. Below is an overview of how these schools of
thought influence the three paradigms-- market liberalism, welfare
capitalism, and post-growth. This overview should enable us to better
understand the many facets of the sustainability debate and develop
approaches that can meet the requirements of a more sustainable future.
However, these classifications are not set in stone and may vary
depending on the specific point of view and context. Some schools of
thought may also fall completely outside the three guiding paradigms. In
pluralist economics, there are a variety of approaches and theories that
can shed light on different aspects of economic and societal design.
Combining and integrating these different perspectives can lead to a
more comprehensive understanding and approach to promoting
sustainability.

\begin{figure}[H]

\centering{

\pandocbounded{\includegraphics[keepaspectratio]{images/Fig5_3_pluralEconParadigms.png}}

}

\caption{\label{fig-pluraleconparad}Our own representation based on the
illustration of Exploring Economics}

\end{figure}%

\section{Green growth and
post-growth}\label{green-growth-and-post-growth}

The two central scientific concepts that shape the debates on
sustainable economics as polar positions and can be located in these
approaches: green growth and
post-growth\textsuperscript{\hyperref[_ftn1]{{[}1{]}}} .

\textbf{Green growth}, as proposed by the OECD (2011) and the World Bank
(2012), is an approach that sees green economic growth as a solution to
ecological challenges. This approach pursues the idea that it is
possible to grow the economy while minimizing environmental impacts. In
our view, this approach can be derived primarily from the welfare
capitalist framework and, to a lesser extent, from the liberal market
framework.

This contrasts with \textbf{post-growth} approaches advocated by
researchers such as Jackson (2009), Seidl and Zahrnt (2010), van den
Bergh (2011) and D'Alisa, Kallis, and Demaria (2016), which are derived
from the post-growth model. These approaches argue that economic growth,
especially as measured by gross domestic product (GDP), and the
associated compulsion to grow are the root causes of ecological
problems. They point out that numerous analyses of the current economic
and social system show that economic growth is unlikely to be a solution
in the current system. This view is based on the recognition that modern
capitalist democracies often depend on permanent increases in profits
and productivity in order to stabilize themselves (Binswanger (2019);
Lessenich (2020); Rosa (2016)), as we have explained under economic
challenges.

The underlying disagreement between these polar positions lies in one of
the most central questions of sustainable economics: Should we focus on
green economic growth to achieve ecological goals, or is it necessary to
rethink the existing economic system and its focus on growth? In the
following section, we will explore these concepts and views in greater
depth in order to develop a better understanding of the challenges and
opportunities in the field of environmental and sustainability policy.

\hyperref[_ftnref1]{{[}1{]}} The term degrowth is also frequently used.
Although the two terms are not entirely synonymous, they largely
overlap. For the sake of simplicity, we will therefore use the term
post-growth in each case. In our view, the strength of the term
post-growth lies primarily in the fact that it does not emphasize a
possible contraction of the economy, but rather the necessary
independence from growth.

\begin{figure}[H]

\centering{

\pandocbounded{\includegraphics[keepaspectratio]{images/Fig5_4_SustEconomics.png}}

}

\caption{\label{fig-sustecon}adapted from Petschow, Lange, and Hofmann
(2020)}

\end{figure}%

\begin{tcolorbox}[enhanced jigsaw, breakable, colbacktitle=quarto-callout-note-color!10!white, left=2mm, bottomtitle=1mm, colback=white, rightrule=.15mm, colframe=quarto-callout-note-color-frame, bottomrule=.15mm, opacityback=0, titlerule=0mm, opacitybacktitle=0.6, toptitle=1mm, toprule=.15mm, title={Further reading}, arc=.35mm, leftrule=.75mm, coltitle=black]

If you are interested, you can watch
\href{https://www.youtube.com/watch?v=2rYNewZEazA&t=1129s}{the statement
by Paolo Gentiloni}(European Commissioner for Economy) at the Beyond
Growth conference to illustrate the green growth position. The
post-growth position is well illustrated in
\href{https://www.youtube.com/watch?v=vij3Q6bE6X4}{a contribution by
Timothée Parrique}at the same conference.

\end{tcolorbox}

\section{Weak and strong
sustainability}\label{weak-and-strong-sustainability}

The two polar positions, green growth and post-growth, refer to
different definitions of sustainability. The concept of sustainability
can be classified on a continuum from weak to strong. In this context,
the extreme positions lead to one-dimensional interpretations that fail
to recognize the multidimensional character of the sustainability model.

\begin{figure}[H]

\centering{

\pandocbounded{\includegraphics[keepaspectratio]{images/Fig5_5_StrongWeakSust.png}}

}

\caption{\label{fig-strongweakSust}adapted from Petschow et al.~(2018:
22)}

\end{figure}%

\section{\texorpdfstring{Weak sustainability -- position of neoclassical
economics\textsuperscript{\hyperref[_ftn1]{{[}1{]}}}}{Weak sustainability -- position of neoclassical economics{[}1{]}}}\label{weak-sustainability-position-of-neoclassical-economics1}

Weak sustainability represents a rather ``lenient'' requirement for
sustainability. For proponents of this view, an action is sustainable if
it offers an overall benefit to the system as a whole or at least does
not reduce its quality. Weak sustainability can therefore be seen as a
kind of basic requirement for sustainability, with the aim of
maintaining a certain level of well-being/quality of life or ensuring
equality. The concept of weak sustainability ``assumes the extensive
and, at least in principle, unlimited {[}\ldots{]} substitutability of
all types of capital'' (Ott and Döring (2004), p.41, own translation)
and is thus based on the premise of neoclassical economics. For example,
natural capital can be replaced by other types of capital; the loss of
biodiversity could be replaced by technological capital. It is assumed
that it is ultimately irrelevant in what physical form the inherited
capital stock is passed on to the next generation -- the only thing that
matters is that the total capital and total utility, and thus the
overall level of welfare, are maintained. The concept of weak
sustainability ties in with neoclassical utility theory, according to
which it is irrelevant how utility is generated. Environmental
economics, which emerged from neoclassical economics, fundamentally
draws on the concept of weak sustainability.

Within the framework of the concept of weak sustainability, a measure
can still be considered sustainable even if it is at the expense of
natural capital. This is the case, for example, if the loss is offset by
an increase in human or physical capital. The extraction of raw
materials such as coal or excessive crude oil production could therefore
be perfectly justifiable. This understanding tends to recognize the
strong role of technological progress in sustainable development.
Evaluating measures based on such a concept is inherently complex, as it
always involves many assumptions. Apart from fundamental questions about
sustainability such as ``What is a good life? How do you measure the
level of prosperity?'', the value of different types of capital must
also be determined. This is because an important prerequisite for
substitutability is that the different types of capital are comparable,
which in practice means that everything must be assigned a monetary
value. Even where there is no market price, a monetary value must
therefore be determined. As early as 1974, William Kapp doubted whether
this monetary valuation could reflect the substantial social value that
is actually at stake K. W. Kapp (1974). In addition, fundamental and
normative questions arise when it comes to operationalization. Which
factors are taken into account? What is the value of natural beauty?
What is the value of a rare bird or a whale? This short
\href{https://www.youtube.com/watch?v=MSxIBYOMQOU}{video} gives some
insight into these discussions.

Criticism of the concept of weak sustainability therefore includes
questions about the unlimited substitutability of natural resources by
reproducible capital. There are also doubts as to whether more goods can
compensate for the loss of environmental quality. Finally, there is
uncertainty about the development of resources and critical thresholds.
The consequence of this criticism is the concept of strong
sustainability.

\textbf{Strong sustainability -- the position of ecological economics}

In contrast to weak sustainability, the position of strong
sustainability considers the value of natural capital to be
irreplaceable. Characteristic of the proponents of strong sustainability
is that their optimism regarding individual substitution possibilities
is much lower. They therefore emphasize the importance of an intact
natural capital stock. They consider it inconceivable that the natural
capital stock could be drastically reduced in favor of another type of
capital. Human-produced capital and natural capital are only
interchangeable to a limited extent. Proponents of strong sustainability
thus take an ecocentric view, in contrast to the anthropocentric view of
weak sustainability.

In contrast to the confidence of neoclassical economists regarding
substitutability, proponents of strong sustainability do not believe in
solutions based on aftercare and reaction, but rather focus on
prevention and anticipation. For example, the increase in the ozone hole
can only be compensated to a limited extent by measures such as
sunscreen, protective clothing, and medical aftercare. Even
technological approaches such as geoengineering are viewed critically,
as they treat the problem superficially rather than addressing its
causes. Geoengineering refers to technical interventions in geochemical
or biogeochemical cycles, for example to slow down climate change or
ocean acidification.

Until now, reducing environmental pollution has mainly focused on
end-of-pipe technologies, such as catalytic converters in cars. At the
same time, long-term environmental problems have been pushed into the
background by such measures as long as their effects have not been
clearly perceived, as in the case of climate change or biodiversity
loss. This seemingly environmental protection is explained by the
concept of \textbf{externalization}, in which environmental and social
costs are shifted to external parties, as described, for example, by
Lessenich (2020). This means that only economic and business costs are
taken into account in pricing, partly because these are easier to
quantify, while the social and environmental costs of providing goods
and services are omitted or shifted to external parties. This leads to
distortions in market prices.

\hyperref[_ftnref1]{{[}1{]}} Excerpt from Pufé (2017), starting on page
105, and
\url{https://thesustainablepeople.com/starke-und-schwache-nachhaltigkeit/}

\begin{longtable}[]{@{}
  >{\raggedright\arraybackslash}p{(\linewidth - 2\tabcolsep) * \real{0.4486}}
  >{\raggedright\arraybackslash}p{(\linewidth - 2\tabcolsep) * \real{0.5514}}@{}}
\caption{Comparison of weak and strong sustainability (Pufé (2017),
p.109)}\label{tbl-weakstrongsust}\tabularnewline
\toprule\noalign{}
\begin{minipage}[b]{\linewidth}\raggedright
weak sustainability
\end{minipage} & \begin{minipage}[b]{\linewidth}\raggedright
strong sustainability
\end{minipage} \\
\midrule\noalign{}
\endfirsthead
\toprule\noalign{}
\begin{minipage}[b]{\linewidth}\raggedright
weak sustainability
\end{minipage} & \begin{minipage}[b]{\linewidth}\raggedright
strong sustainability
\end{minipage} \\
\midrule\noalign{}
\endhead
\bottomrule\noalign{}
\endlastfoot
\begin{minipage}[t]{\linewidth}\raggedright
\begin{itemize}
\item
  human-centered perspective (anthropo-centrism)
\item
  based on neoclassical economics
\item
  assumes harmony between growth and the environment
\item
  substitutability of natural capital
\item
  growth paradigm, growth is central
\item
  optimistic about the potential for continued growth
\end{itemize}

strategy: rely on technological efficiency, economic growth and market
mechanism

conventional cost-benefit analysis as primary decision-making tool
\end{minipage} & \begin{minipage}[t]{\linewidth}\raggedright
\begin{itemize}
\item
  nature-centered perspective (ecocentrism)
\item
  based on ecological economics
\item
  conflict between growth and environment
\item
  no substitutability of natural capital
\item
  impossibility of infinite growth
\item
  sceptical about continued growth
\end{itemize}

strategy: stop/limit growth, ethical restraint and efficiency at both
personal and political levels

sees traditional cost-benefit analysis as inadequate for sustainability
issues
\end{minipage} \\
\end{longtable}

As already mentioned at the beginning, the polar positions of
sustainable economics, green growth, and post-growth are based on
different understandings of sustainability, are supported by scientific
concepts of different schools of thought and are accordingly oriented
toward different economic policy frameworks. In the next section, we
will take a closer look at the range of approaches and their respective
focuses, concentrating on the post-growth framework. The reason for
focusing on post-growth is that we believe this economic policy
framework makes a key contribution to a sustainable economy.

\bookmarksetup{startatroot}

\chapter{Approaches of sustainable
economics}\label{approaches-of-sustainable-economics}

As outlined in the preceding section, the debates around sustainable
economics are shaped by different guiding frameworks, which are
influenced by various theoretical perspectives within pluralist
economics and differ in their understanding of sustainability and their
respective viewpoints. Most approaches within sustainable economics can
be assigned to the post-growth framework. From the welfare capitalist
framework, approaches such as green growth and the circular economy can
be derived. The liberal market framework lies largely outside the debate
of sustainable economics, although certain aspects of green growth are
discussed, such as the internalization of external effects.

In the following, we will engage more deeply with specific approaches.
Since the debates within sustainable economics are very diverse, no
definitive or fixed classification of the approaches can be made.
Nevertheless, we have distinguished between several approaches here.
While there are many overlaps among them, in our view they can still be
differentiated based on their distinct emphases. This classification is
intended to serve as an orientation. For the approaches within the
post-growth framework, we follow Schmelzer and Vetter (2023).

\textbf{Learning Outcome}

By the end of the module, students\ldots{}

\begin{itemize}
\tightlist
\item
  are able to distinguish between the main approaches of sustainable
  economics.
\item
  know the key characteristics, solutions, and concepts of one approach
  and are able to explain it using examples.
\end{itemize}

\section{Approaches of sustainable
economics}\label{approaches-of-sustainable-economics-1}

Below you will find a list of different approaches within sustainable
economics and corresponding material for each approach. Please choose
two of the following approaches. Work through the material for each
approach.
\href{https://app.mural.co/t/cde6674/m/cde6674/1759743384617/4b38e44dea858b12fbb1cca4bc38b88505fed944?sender=ua98dc58f8437b9b336bb4662}{On
Mural} you will find question regarding the approaches. Try to answer
the questions for the approaches you have chosen. This way you can gain
an overview of the different approaches based on the work of other
students. The lecturers will regularly check the Mural board and delete
statements that are strongly incorrect or misleading, however, the
lecturers will not correct every answer provided by the students.

\subsection{Welfare capitalist
paradigm}\label{welfare-capitalist-paradigm-1}

\subsubsection{Green Growth}

\href{https://ilias.unibe.ch/go/file/3512925/download}{The text by
Petschow, Lange, and Hofmann (2020)} serves as an introduction and
provides an overview of Green Growth.
\href{https://tobira.unibe.ch/!v/B96EQrQRZHm}{This video snippet} is
from a contribution by Michael Jacobs for the panel ``Focus Panel 1 --
Which Prosperous Future? Confronting Narratives of Growth'' at the
Beyond Growth Conference 2023. If interested,
\href{https://www.youtube.com/watch?v=KqtlQcU3PaQ}{the full panel
discussion} can be watched.

\subsubsection{Circular Economy}

This video serves as a brief introduction.
\href{https://ilias.unibe.ch/go/file/3512927/download}{The text by
Lazarevic and Brandão (2020)} provides a more in-depth introduction to
the topic. If you are interested, the text by
\href{https://ilias.unibe.ch/go/file/3512928/download}{Bocken et
al.~(2022)} can be read as an additional resource.

\url{https://www.youtube.com/watch?v=NBEvJwTxs4w}

\subsection{Post-growth paradigm}\label{post-growth-paradigm}

\subsubsection{Growth-independent institutions}

\href{https://ilias.unibe.ch/go/file/3512933/download}{The paper by
Corlet Walker et al.~(2024)} presents an approach to the question of the
UK welfare system. And the following video presents a further discussion
on making the UK welfare system more growth independent.

\url{https://www.youtube.com/watch?v=gK4iuguEDgQ}

\subsubsection{Sufficiency-oriented way of living}

This video by Niko Paech, prominent proponent of this approach, will
provide an introduction into the main ideas of this approach.

\url{https://www.youtube.com/watch?v=rS3ldLZ_kYE}

\href{https://ilias.unibe.ch/go/file/3512937/download}{The text by
Juliet Schor (2008)}, a pioneer in the research on consumerism in the
US, provides a further discussion of this approach. The text is
relatively old, however, provides a good overview of the central issues.
However, especially with regard to work time reduction, there as been a
lot of new research since then, you are encouraged to explore if you are
interested in.

\url{https://www.youtube.com/watch?v=H3eSFIu3D0E}

\subsubsection{Collective action}

\href{https://ilias.unibe.ch/go/file/3512934/download}{The paper by
David Bollier (2015)} gives a good sense of the core idea of such
approaches and presents a good overview. This video on commoning the
city provides and discusses more concrete ideas of how this approach
might look like:

\url{https://www.youtube.com/watch?v=qCmJcPkpI98}

\url{https://www.youtube.com/watch?v=4MKu_jB-UkM}

\url{https://www.youtube.com/watch?v=gydkOrGrs8Q}

\subsubsection{Overcoming the imperial way of life}

\href{https://ilias.unibe.ch/go/file/3512940/download}{The text form the
ILA collective} provides a concise introduction into the idea of this
approach. The video taken from a talk by Ulrich Brand, one of the main
proponents of this approach, offers further discussions on this
approach:

\url{https://www.youtube.com/watch?v=4UiT5J0HPHc}

\bookmarksetup{startatroot}

\chapter*{References}\label{references}
\addcontentsline{toc}{chapter}{References}

\markboth{References}{References}

\phantomsection\label{refs}
\begin{CSLReferences}{1}{0}
\bibitem[\citeproctext]{ref-althouseEcologicallyUnequalExchange2023}
Althouse, Jeffrey, Louison Cahen-Fourot, Bruno Carballa-Smichowski,
Cédric Durand, and Steven Knauss. 2023. {``Ecologically Unequal Exchange
and Uneven Development Patterns Along Global Value Chains.''}
\emph{World Development} 170 (October): 106308.
\url{https://doi.org/10.1016/j.worlddev.2023.106308}.

\bibitem[\citeproctext]{ref-anievasHowWestCame2015a}
Anievas, Alexander, and Kerem Nişancioğlu. 2015. \emph{How the {West
Came} to {Rule}: {The Geopolitical Origins} of {Capitalism}}. London:
Pluto Press. \url{https://doi.org/10.2307/j.ctt183pb6f}.

\bibitem[\citeproctext]{ref-vandenberghEnvironmentGrowthCriticism2011}
Bergh, Jeroen C. J. M. van den. 2011. {``Environment Versus Growth ---
{A} Criticism of {`Degrowth'} and a Plea for {`a-Growth'}.''}
\emph{Ecological Economics} 70 (5): 881--90.
\url{https://doi.org/10.1016/j.ecolecon.2010.09.035}.

\bibitem[\citeproctext]{ref-binswangerWachstumszwangWarumVolkswirtschaft2019}
Binswanger, Mathias. 2019. \emph{Der Wachstumszwang: warum die
Volkswirtschaft immer weiterwachsen muss, selbst wenn wir genug haben}.
Weinheim: Wiley-VCH Verlag GmbH \& Co. KGaA.

\bibitem[\citeproctext]{ref-bontrupVolkswirtschaftslehreAusOrthodoxer2021a}
Bontrup, Heinz-J., and Ralf-M. Marquardt. 2021.
\emph{Volkswirtschaftslehre Aus Orthodoxer Und Heterodoxer {Sicht}}. De
Gruyter. \url{https://doi.org/10.1515/9783110619379}.

\bibitem[\citeproctext]{ref-carballasmichowskiParticipationGlobalValue2021}
Carballa Smichowski, Bruno, Cédric Durand, and Steven Knauss. 2021.
{``Participation in Global Value Chains and Varieties of Development
Patterns.''} \emph{Cambridge Journal of Economics} 45 (2): 271--94.
\url{https://doi.org/10.1093/cje/beaa046}.

\bibitem[\citeproctext]{ref-ceasarPlanetaryHealthCheck2024}
Ceasar, L., B. Sakschweski, L. S. Andersen, T. Beringer, J. Braun, D.
Dennis, D. Gerten, et al. 2024. {``Planetary {Health Check Report}
2024.''} Potsdam: Potsdam Institute for Climate Impact Research.

\bibitem[\citeproctext]{ref-changKickingAwayLadder2002}
Chang, Ha-Joon. 2002. \emph{Kicking {Away} the {Ladder}: {Development
Strategy} in {Historical Perspective}}. 1st ed. Anthem {Studies} in
{Development} and {Globalization}. London: Anthem Press.

\bibitem[\citeproctext]{ref-changLifeUniverseEverything2014}
---------. 2014. {``Life, the {Universe} and {Everything}. {What Is
Economics}?''} In \emph{Economics: {The User}'s {Guide}}, by Ha-Joon
Chang, First U.S. edition, 13--22. New York: Bloomsbury Press.

\bibitem[\citeproctext]{ref-coaseProblemSocialCost1960}
Coase, Ronald H. 1960. {``The {Problem} of {Social Cost}.''} \emph{The
Journal of Law \& Economics} 3: 1--44.

\bibitem[\citeproctext]{ref-dalisaNachwortAusteritaetZur2016}
D'Alisa, Giacoma, Giorgos Kallis, and Federico Demaria. 2016.
{``Nachwort: Von der Austerität zur Dépense.''} In \emph{Degrowth:
Handbuch für eine neue Ära}, edited by Giacoma D'Alisa, Federico
Demaria, and Giorgios Kallis, 297--86. Berlin: Oekom Verlag.

\bibitem[\citeproctext]{ref-davisEconomistsOddStand2016}
Davis, John B. 2016. {``Economists' {Odd Stand} on the
{Positive}--{Normative Distinction}: {A Behavioral Economics View}.''}
In \emph{The {Oxford Handbook} of {Professional Economic Ethics}}, by
John B. Davis, edited by George F. DeMartino and Deirdre McCloskey,
199--218. Oxford University Press.
\url{https://doi.org/10.1093/oxfordhb/9780199766635.013.011}.

\bibitem[\citeproctext]{ref-davisLateVictorianHolocausts2017}
Davis, Mike. 2017. \emph{Late {Victorian} Holocausts: {El Niño} Famines
and the Making of the {Third World}}. Paperback edition. London New
York: Verso.

\bibitem[\citeproctext]{ref-dekeyzerInclusiveCommonsSustainability2018}
De Keyzer, Maïka. 2018. \emph{Inclusive {Commons} and the
{Sustainability} of {Peasant Communities} in the {Medieval Low
Countries}}. 1st Edition: Routledge.

\bibitem[\citeproctext]{ref-demoorDilemmaCommonersUnderstanding2015}
De Moor, Tine. 2015. \emph{The {Dilemma} of the {Commoners}:
{Understanding} the {Use} of {Common-Pool Resources} in {Long-Term
Perspective}}. Political {Economy} of {Institutions} and {Decisions}.
Cambridge: Cambridge University Press.
\url{https://doi.org/10.1017/CBO9781139135450}.

\bibitem[\citeproctext]{ref-degenArbeitUndKapital2012}
Degen, Bernard. 2012. {``Arbeit Und {Kapital}.''} In
\emph{Wirtschaftsgeschichte Der {Schweiz} Im 20. {Jahrhundert}}, edited
by Patrick Halbeisen, Margrit Müller, and Béatrice Veyrassat, 872--922.
Basel: Schwabe. \url{http://edoc.unibas.ch/dok/A6070259}.

\bibitem[\citeproctext]{ref-dorningerGlobalPatternsEcologically2021a}
Dorninger, Christian, Alf Hornborg, David J. Abson, Henrik Von Wehrden,
Anke Schaffartzik, Stefan Giljum, John-Oliver Engler, Robert L. Feller,
Klaus Hubacek, and Hanspeter Wieland. 2021. {``Global Patterns of
Ecologically Unequal Exchange: {Implications} for Sustainability in the
21st Century.''} \emph{Ecological Economics} 179 (January): 106824.
\url{https://doi.org/10.1016/j.ecolecon.2020.106824}.

\bibitem[\citeproctext]{ref-durandIntellectualMonopolyGlobal2020}
Durand, Cédric, and Wiliiam Milberg. 2020. {``Intellectual Monopoly in
Global Value Chains.''} \emph{Review of International Political Economy}
27 (2): 404--29. \url{https://doi.org/10.1080/09692290.2019.1660703}.

\bibitem[\citeproctext]{ref-engelkampEinfuehrungVolkswirtschaftslehre2013a}
Engelkamp, Paul, and Friedrich L. Sell. 2013. \emph{Einführung in die
Volkswirtschaftslehre}. Springer-Lehrbuch. Berlin, Heidelberg: Springer
Berlin Heidelberg. \url{https://doi.org/10.1007/978-3-642-36522-5}.

\bibitem[\citeproctext]{ref-friedliBetriebswirtschaftslehreZusammenhangeVerstehen2019}
Friedli, Vera, Renato Müller Vasquez Callo, and Rahel Balmer-Zahnd.
2019. \emph{Betriebswirtschaftslehre: {Zusammenhänge} Verstehen}. 4.
Auflage. Hep Verlag AG.

\bibitem[\citeproctext]{ref-ghoshNutmegsCurseParables2021}
Ghosh, Amitav. 2021. \emph{The Nutmeg's Curse: Parables for a Planet in
Crisis}. Chicago: University of Chicago Press.

\bibitem[\citeproctext]{ref-hickelDivideBriefGuide2017}
Hickel, Jason. 2017. \emph{The {Divide}: {A Brief Guide} to {Global
Inequality} and {Its Solutions}}. London: William Heinemann.

\bibitem[\citeproctext]{ref-hillAdamSmithThumos2012a}
Hill, Lisa. 2012. {``Adam {Smith} on {{\emph{Thumos}}} and Irrational
Economic {`Man'}.''} \emph{The European Journal of the History of
Economic Thought} 19 (1): 1--22.
\url{https://doi.org/10.1080/09672561003632550}.

\bibitem[\citeproctext]{ref-huberFuelingCapitalismOil2013}
Huber, Matt. 2013. {``Fueling {Capitalism}: {Oil}, the {Regulation
Approach}, and the {Ecology} of {Capital}.''} \emph{Economic Geography}
89 (2): 171--94. \url{https://doi.org/10.1111/ecge.12006}.

\bibitem[\citeproctext]{ref-jacksonProsperityGrowthTransition2009}
Jackson, Tim. 2009. \emph{Prosperity Without Growth? {The} Transition to
a Sustainable Economy}. New York: Sustainable Development Commission.

\bibitem[\citeproctext]{ref-jarrigeContaminationEarthHistory2020}
Jarrige, François, and Thomas Le Roux. 2020. \emph{The {Contamination}
of the {Earth}: {A History} of {Pollutions} in the {Industrial Age}}.
Translated by Janice Egan and Michael Egan. History for a Sustainable
Future. Cambridge, Massachusetts London: The MIT Press.

\bibitem[\citeproctext]{ref-jevonsTheoryPoliticalEconomy1879}
Jevons, William Stanley. 1879. \emph{The {Theory} of {Political
Economy}}. 2nd ed. London: Macmillan.

\bibitem[\citeproctext]{ref-kallisOilEconomySystematic2017}
Kallis, Giorgos, and Jalel Sager. 2017. {``Oil and the Economy: {A}
Systematic Review of the Literature for Ecological Economists.''}
\emph{Ecological Economics} 131 (January): 561--71.
\url{https://doi.org/10.1016/j.ecolecon.2016.08.011}.

\bibitem[\citeproctext]{ref-kappEnvironmentalPoliciesDevelopment1974}
Kapp, K. William. 1974. \emph{Environmental {Policies} and {Development
Planning} in {Contemporary China} and {Other Essays}}. De Gruyter.
\url{https://doi.org/10.1515/9783112317273}.

\bibitem[\citeproctext]{ref-kappSocialCostsPrivate1950}
Kapp, William. 1950. \emph{The {Social Costs} of {Private Enterprise}}.
Cambridge, Mass: Harvard University Press.

\bibitem[\citeproctext]{ref-keenDebunkingEconomicsNaked2010}
Keen, Steve. 2010. \emph{Debunking Economics: The Naked Emperor of the
Social Sciences}. 5. impression. London: Zed Books.

\bibitem[\citeproctext]{ref-kumhofLeveragingInequalityLong2010}
Kumhof, Michael, and Romain Rancière. 2010. {``Leveraging {Inequality}:
{Long} Periods of Unequal Incomes Spur Borrowing from the Rich,
Increasing the Risk of Major Economic Crises.''} \emph{Finance \&
Development}, December 2010.
\url{https://www.imf.org/external/pubs/ft/fandd/2010/12/Kumhof.htm}.

\bibitem[\citeproctext]{ref-lessenichNebenUnsSintflut2020}
Lessenich, Stephan. 2020. \emph{Neben uns die Sintflut: die
Externalisierungsgesellschaft und ihr Preis}. 6. Auflage. München:
Hanser Berlin.

\bibitem[\citeproctext]{ref-malmFossilCapitalRise2016}
Malm, Andreas. 2016. \emph{Fossil Capital: The Rise of Steam-Power and
the Roots of Global Warming}. London: Verso.

\bibitem[\citeproctext]{ref-marshallIndustryTradeStudy1919}
Marshall, Alfred. 1919. \emph{Industry and {Trade}: {A Study} of
{Industrial Technique} and {Business Organization}; and of {Their
Influences} on the {Conditions} of {Various Classes} and {Nations}}.
Macmillan.

\bibitem[\citeproctext]{ref-millDefinitionPoliticalEconomy1996a}
Mill, John Stuart. 1996. {``On the {Definition} of {Political Economy};
and on the {Method} of {Phlosophical Investigation} in That
{Science}.''} In \emph{Essays on {Economics} and {Scoiety}, {Collected
Works} of {John Stuart Mill}}, edited by John M. Robson. University of
Toronto Press.

\bibitem[\citeproctext]{ref-minskyFinancialInstabilityHypothesis1992}
Minsky, Hyman. 1992. {``The {Financial Instability Hypothesis}.''}
Working Paper No. 74. Annandale-on-Hudson, NY: The Jerome Levy Economics
Institute of Bard College.

\bibitem[\citeproctext]{ref-mirowskiMoreHeatLight1989a}
Mirowski, Philip. 1989. \emph{More {Heat} Than {Light}: {Economics} as
{Social Physics}, {Physics} as {Nature}'s {Economics}}. 1st ed.
Cambridge University Press.
\url{https://doi.org/10.1017/CBO9780511559990}.

\bibitem[\citeproctext]{ref-morganCharacterRationalEconomic1996}
Morgan, Mary S. 1996. {``The {Character} of {`{Rational Economic
Man}'}.''} Working Paper. London School of Economics, Department of
Economic History.
\url{https://www.researchgate.net/publication/254410469_The_Character_of_'Rational_Economic_Man'}.

\bibitem[\citeproctext]{ref-morganEconomicManModel2006a}
---------. 2006. {``Economic {Man} as {Model Man}: {Ideal Types},
{Idealization} and {Caricatures}.''} \emph{Journal of the History of
Economic Thought} 28 (1): 1--27.
\url{https://doi.org/10.1080/10427710500509763}.

\bibitem[\citeproctext]{ref-morganWorldModelHow2012a}
---------. 2012. \emph{The {World} in the {Model}: {How Economists Work}
and {Think}}. 1st ed. Cambridge University Press.
\url{https://doi.org/10.1017/CBO9781139026185}.

\bibitem[\citeproctext]{ref-nelsonStudyChoiceStudy1993}
Nelson, Julie A. 1993. {``The Study of Choice or the Study of
Provisioning?''} In \emph{Beyond Economic Man: Feminist Theory and
Economics}, edited by Marianne A. Ferber and Julie A. Nelson, 23--37.
Chicago: The University of Chicago Press.

\bibitem[\citeproctext]{ref-nordhausClimateChangeUltimate2019}
Nordhaus, William. 2019. {``Climate {Change}: {The Ultimate Challenge}
for {Economics}.''} \emph{American Economic Review} 109 (6): 1991--2014.
\url{https://doi.org/10.1257/aer.109.6.1991}.

\bibitem[\citeproctext]{ref-novyZukunftsfaehigesWirtschaftenHerausforderungen2023}
Novy, Andreas, Richard Bärnthaler, and Magdalena Prieler. 2023.
\emph{Zukunftsfähiges Wirtschaften: Herausforderungen der
sozialökologischen Transformation}. 2. Auflage. Arbeitsgesellschaft im
Wandel. Weinheim: Juventa Verlag.

\bibitem[\citeproctext]{ref-oecdGreenGrowth2011}
OECD. 2011. {``Towards {Green Growth}.''} Paris: OECD Publishing.
\url{https://doi.org/10.1787/9789264111318-en}.

\bibitem[\citeproctext]{ref-ottTheorieUndPraxis2004}
Ott, Konrad, and Ralf Döring. 2004. \emph{Theorie Und {Praxis} Starker
{Nachhaltigkeit}}. Marburg: Metropolis.

\bibitem[\citeproctext]{ref-paechBefreiungVomUberfluss2012}
Paech, Niko. 2012. \emph{Befreiung Vom {Überfluss}. {Auf} Dem {Weg} in
Die {Postwachstumsökonomie}}. München: Oekom Verlag.

\bibitem[\citeproctext]{ref-palmaLatinAmericanEconomies2003}
Palma, Jose Gabriel. 2003. {``The {Latin American Economies During} the
{Second Half} of the {Twentieth Century} - from the {Age} Og '{ISI}' to
the {Age} of '{The End} of {History}'.''} In \emph{Rethinking
{Development Economics}}, edited by Ha-Joon Chang, 125--52. Anthem
{Frontiers} of {Global Political Economy} and {Development}. London:
Anthem Press.

\bibitem[\citeproctext]{ref-paretoAnwendungenMathematikAuf1902}
Pareto, Vilfredo. 1902. {``Anwendungen Der {Mathematik} Auf
{Nationalokonomie}.''} In \emph{Encyklopädie Der Mathematischen
{Wissenschaften} Mit {Einschluss} Ihrer {Anwendungen}}, edited by
Wilhelm Franz Meyer, Band 1 Teil 2:1094--1120. Leipzig: B.G. Teubner.

\bibitem[\citeproctext]{ref-petschowSocialWellBeingPlanetary2020}
Petschow, Ulrich, Steffen Lange, and David Hofmann. 2020. {``Social
{Well-Being Within Planetary Boundaries}: The {Precautionary Post-Growth
Approach}.''} Umweltbundesamt.
\url{https://www.umweltbundesamt.de/en/publikationen/social-well-being-within-planetary-boundaries-the}.

\bibitem[\citeproctext]{ref-pirgmaierNeoclassicalTrojanHorse2017}
Pirgmaier, Elke. 2017. {``The {Neoclassical Trojan Horse} of
{Steady-State Economics}.''} \emph{Ecological Economics} 133 (March):
52--61. \url{https://doi.org/10.1016/j.ecolecon.2016.11.010}.

\bibitem[\citeproctext]{ref-polanyiLivelihoodMan1977}
Polanyi, Karl. 1977. \emph{The {Livelihood} of {Man}}. Edited by Harry
W. Pearson. Studies in Social Discontinuity. New York: Academic Press.

\bibitem[\citeproctext]{ref-polanyiGreatTransformation2017}
---------. 2017. \emph{The {Great Transformation}}. 13. Auflage.
Sinzheim: Suhrkamp Taschenbuch Verlag.

\bibitem[\citeproctext]{ref-ponteHiddenCostsEnvironmental2022}
Ponte, Stefano. 2022. {``The Hidden Costs of Environmental Upgrading in
Global Value Chains.''} \emph{Review of International Political Economy}
29 (3): 818--43. \url{https://doi.org/10.1080/09692290.2020.1816199}.

\bibitem[\citeproctext]{ref-pufeNachhaltigkeit2017}
Pufé, Iris. 2017. \emph{Nachhaltigkeit}. 3., überarbeitete und
erweiterte Auflage. Konstanz: UVK Verlagsgesellschaft mbH.

\bibitem[\citeproctext]{ref-putnamCollapseFactValue2002}
Putnam, Hilary. 2002. \emph{The Collapse of the Fact/ Value Dichotomy
and Other Essays}. Cambridge, MA: Harvard University Press.

\bibitem[\citeproctext]{ref-reissMathematicsEconomicsSchmoller2000a}
Reiss, Julian. 2000. {``Mathematics in Economics: {Schmoller}, {Menger}
and {Jevons}.''} \emph{Journal of Economic Studies} 27 (4/5): 477--91.
\url{https://doi.org/10.1108/01443580010342393}.

\bibitem[\citeproctext]{ref-robbinsEssayNatureSignificance1932}
Robbins, Lionel. 1932. \emph{An {Essay} on the {Nature} and
{Significance} of {Economic Science}}. London: Macmillan \& Co. Limited.

\bibitem[\citeproctext]{ref-rosaSocialAccelerationEthical2003}
Rosa, Hartmut. 2003. {``Social {Acceleration}: {Ethical} and {Political
Consequences} of a {Desynchronized High}--{Speed Society}.''}
\emph{Constellations} 10 (1): 3--33.
\url{https://doi.org/10.1111/1467-8675.00309}.

\bibitem[\citeproctext]{ref-rosaBeschleunigungUndEntfremdung2016}
---------. 2016. \emph{Beschleunigung und Entfremdung: Entwurf einer
kritischen Theorie spätmoderner Zeitlichkeit}. Translated by Robin
Celikates. 5. Auflage. Berlin: Suhrkamp.

\bibitem[\citeproctext]{ref-saaveEinverleibenUndExternalisieren2022}
Saave, Anna. 2022. \emph{Einverleiben und Externalisieren: zur
Innen-Außen-Beziehung der kapitalistischen Produktionsweise}.
Sozialtheorie. Bielefeld: transcript.
\url{https://doi.org/10.1515/9783839458341}.

\bibitem[\citeproctext]{ref-santariusReboundEffektBlinderFleck2014}
Santarius, Tilman. 2014. {``Der {Rebound-Effekt}: Ein Blinder {Fleck}
Der Sozial-Ökologischen {GesellschaftstransformationRebound Effects}:
{Blind Spots} in the {Socio-Ecological Transition} of {Industrial
Societies}.''} \emph{GAIA - Ecological Perspectives for Science and
Society} 23 (2): 109--17. \url{https://doi.org/10.14512/gaia.23.2.8}.

\bibitem[\citeproctext]{ref-schauppStoffwechselpolitikArbeitNatur2024}
Schaupp, Simon. 2024. \emph{Stoffwechselpolitik: Arbeit, Natur und die
Zukunft des Planeten}. Erste Auflage, Sonderdruck, Originalausgabe.
edition suhrkamp. Berlin: Suhrkamp.

\bibitem[\citeproctext]{ref-schmelzerDegrowthPostwachstum2019}
Schmelzer, Matthias, and Andrea Vetter. 2019.
\emph{Degrowth/{Postwachstum}}. Hamburg: Junius.

\bibitem[\citeproctext]{ref-schmelzerDegrowthPostwachstumZur2023}
---------. 2023. \emph{Degrowth/Postwachstum zur Einführung}. 3.,
unveränderte Auflage. Hamburg: Junius.

\bibitem[\citeproctext]{ref-sedlacekOekonomieGutUnd2012}
Sedláček, Tomáš. 2012. \emph{Die Ökonomie von Gut und Böse}. Translated
by Ingrid Pross-Gill. München: Hanser.

\bibitem[\citeproctext]{ref-seidlPostwachstumsgesellschaftKonzepteFur2010}
Seidl, Irmi, and Angelika Zahrnt, eds. 2010.
\emph{Postwachstumsgesellschaft. {Konzepte} Für Die {Zukunft}}. Ökologie
Und {Wirtschaftsforschung}. Marbug.

\bibitem[\citeproctext]{ref-steffenTrajectoryAnthropoceneGreat2015}
Steffen, Will, Wendy Broadgate, Lisa Deutsch, Owen Gaffney, and Cornelia
Ludwig. 2015. {``The Trajectory of the {Anthropocene}: {The Great
Acceleration}.''} \emph{The Anthropocene Review} 2 (1): 81--98.
\url{https://doi.org/10.1177/2053019614564785}.

\bibitem[\citeproctext]{ref-vazquezMarshallMathematizationEconomics1995a}
Vazquez, Andres. 1995. {``Marshall and the {Mathematization} of
{Economics}.''} \emph{Journal of the History of Economic Thought} 17
(2): 247--65. \url{https://doi.org/10.1017/S1053837200002625}.

\bibitem[\citeproctext]{ref-weizsaeckerFaktorVierDoppelter1997}
Weizsäcker, Ernst Ulrich von, Amory Lovins, and L. Hunter Lovins. 1997.
\emph{Faktor Vier - doppelter Wohlstand - halbierter Naturverbrauch. Der
neue Bericht an den Club of Rome.} 10. Aufl. München: Droemer Knaur.

\bibitem[\citeproctext]{ref-wilsonHistoryHomoEconomicus2014}
Wilson, David, and William Dixon. 2014. \emph{A {History} of {Homo
Economicus} - {The Nature} of the {Moral} in {Economic Theory}}.
Paperback Edition. Routledge {Studies} in the {History} of {Economics}.
London; New York: Routledge.

\bibitem[\citeproctext]{ref-worldbankInclusiveGreenGrowth2012}
World Bank. 2012. \emph{Inclusive {Green Growth}: {The Pathway} to
{Sustainable Development}}.
\url{https://doi.org/10.1596/978-0-8213-9551-6}.

\end{CSLReferences}


\backmatter


\end{document}
